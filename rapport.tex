\documentclass[a4paper,10pt]{article}
\usepackage[utf8]{inputenc}
\usepackage[T1]{fontenc}
\usepackage[french]{babel}
\usepackage{amsmath,amssymb,amsthm}
\usepackage{mathrsfs}
\usepackage{graphicx}
\usepackage{tikz}
%\usepackage{listings}
%\usepackage{xcolor}

%opening
\title{Théorème de Gabriel}
\author{Maxime Buron, Édouard Rousseau\\
Travail encadré par Pierre-Guy Plamondon}

%Création des labels Théorème, Lemme, etc
\newtheorem{thm}{Théorème}[section]
\newtheorem{lm}{Lemme}[section]
\newtheorem{defi}{Définition}[section]
\newtheorem{prop}{Proposition}[section]
\newtheorem{ex}{Exemple}[section]

%Feuilles de styles Tikz
\tikzset{sommet/.style={circle, draw, fill,scale=0.4}}
\tikzset{fleche/.style={->,>=latex}}


%Fin des paramètres globaux et début du document.
\begin{document}

\maketitle

%Le résumé de l'article
\begin{abstract} 
	Ce mémoire a été écrit dans le cadre des Travaux Encadrés de Recherche (TER) au cours du M1 Mathématiques Fondamentales et Appliquées (MFA) de l'université Paris-Sud. Nous y démontrons un théorème dû à Pierre Gabriel datant de 1972, cependant, la preuve utilisée ici n'est pas la preuve orginale de Gabriel, nous suivons la preuve de Bernstein, Gel'Fand et Ponomarev issue de \cite{BGP72}.
\end{abstract}

\tableofcontents

\clearpage
        
\section{Notions préliminaires}
Afin de comprendre l'énoncé, puis la preuve, du théorème de Gabriel, nous avons besoin de notions. Dans toute cette partie, nous introduirons les concepts nécessaires à la bonne compréhension du lecteur.

\subsection{Carquois}
Le théorème de Gabriel donne de très beaux résultats sur les représentations de carquois. Cette notion est donc primordiale pour toute la suite de ce document.
\begin{defi}[Carquois]
	\label{carquois}
	Un carquois est un graphe orienté, où les boucles et les arêtes multiples sont autorisées.
\end{defi}
On considérera seulement des carquois connexes dans les exemples ci dessous.
	\begin{ex}
		\begin{tikzpicture}
			\node[sommet] (A) at (0,0) {};
			\node[sommet] (B) at (2,0) {};
			\draw[fleche] (A) to (B);
		\end{tikzpicture}
	\end{ex}
	\begin{ex}
		\begin{tikzpicture}
			\node[sommet] (A) at (0,0) {};
			\node[sommet] (B) at (2,0) {};
			\draw[fleche] (A) to[bend left] (B);
			\draw[fleche] (B) to[bend left] (A);
			\draw[fleche] (A.north) arc (0:348:.5);
		\end{tikzpicture}
	\end{ex}
	\begin{ex}
		\begin{tikzpicture}
			\node[sommet] (A) at (0,0) {};
			\node[sommet] (B) at (4,0) {};
			\node[sommet] (C) at (2,1) {};
			\node[sommet] (D) at (4,2) {};
			\node[sommet] (E) at (0,2) {};
			\draw[fleche] (A) to (C);
			\draw[fleche] (A) to (B);
			\draw[fleche] (E) to (A); 
			\draw[fleche] (C) to[bend left] (D);
			\draw[fleche] (C) to[bend right] (D);
			\draw[fleche] (E) to[bend left] (D);
			\draw[fleche] (B) to (C);
			\draw[fleche] (C) to (E);
			\draw[fleche] (B) to[bend left] (A);
			\draw[fleche] (E) to[bend left] (A);
			\draw[fleche] (A) to[bend left] (E);
		\end{tikzpicture}
	\end{ex}


\subsection{Représentations de carquois}
Maintenant que nous connaissons la définition d'un carquois, et que nous pouvons nous représenter l'objet, nous sommes capables d'aborder le coeur du sujet : les représentations de carquois. À l'issue de cette partie, nous serons même capables d'énoncer, et comprendre, une partie du théorème de Gabriel.

\subsection{Théorie des catégories}
Avant de traiter la preuve du thérorème de Gabriel, nous introduisons le langage de la théorie des catégories, celui nous aidera au court de la preuve. Les notations utilisées ici sont celle de \cite{A97}, et de plus amples détails sur les catégories peuvent y être trouvées, nous ne donnons ici qu'une brève présentation des notions qui nous serons utiles par la suite.

\subsection{Système de racines}
Afin de faire le lien entre les diagrammes de Dynkin et les représentations de carquois, il nous sera utile de savoir ce qu'est un système de racines, l'arrivée des diagrammes de Dynkin et des groupes de Weil nous permettra même de préciser le théorème de Gabriel que nous connaissions jusqu'à maintenant.



\clearpage
\section{Le théorème de Gabriel}
Nous sommes désormais capables d'énoncer la version finale du théorème de Gabriel :
	\begin{thm}
		On a, dans tout espace de Hilbert $\mathbb{H}$ : $e^{i\pi}+1=0$.
	\end{thm}
	\begin{proof}
		La preuve est élémentaire.
	\end{proof}	

\subsection{Foncteur image et foncteur de Coxeter}
Soit $\Gamma$ un ensemble de sommets et $\Lambda$ une orientation de $\Gamma$. On se place dans la catégorie $\mathcal{L}(\Gamma,\Lambda)$. Nous allons construire des foncteurs images, associés aux sommets $\alpha \in \Gamma_{0}$ de $\Gamma$ pour lesquels la direction des flêches le contenant est tout le temps la même. On note $\Gamma^{\alpha}=\{ l\in \Gamma_{1} : s(l)=\alpha\text{ ou }t(s)=\alpha\}$ l'ensemble des arêtes contenant $\alpha$.
\begin{defi}
	On dit qu'un sommet $\alpha \in \Gamma_{0}$ est un puits, ou qu'il est (+)-accessible, si $\forall l \in \Gamma^{\alpha},\; t(s)=\alpha$, c'est-à-dire si toutes les arêtes contenant $\alpha$ arrivent en $\alpha$ et qu'il n'y a pas de boucles de $\alpha$ dans lui même.

On dit de même qu'un sommet $\alpha \in \Gamma_{0}$ est une source, ou qu'il est (-)-accessible, si $\forall l \in \Gamma^{\alpha},\; s(s)=\alpha$, c'est-à-dire si toutes les arêtes contenant $\alpha$ partent de $\alpha$ et qu'il n'y a pas de boucles de $\alpha$ dans lui même.
\end{defi}
Supposons que le sommet $\alpha \in \Gamma_{0}$ soit un puits avec l'orientation $\Lambda$. On note $\sigma_{\alpha}\Lambda$ l'orientation obtenue à partir de $\Lambda$ en inversant le sens des flêches contenant $\alpha$. Nous allons construire une application qui à un objet $(V,f)$ de $\mathcal{L}(\Gamma,\Lambda)$ associe un objet $(W,g)$ de $\mathcal{L}(\Gamma,\sigma_{\alpha}\Lambda)$.
\begin{defi}
	On notera cette application $F_{\alpha}^{+}$ et on aura $F_{\alpha}^{+}(V,f)=(W,g)$. On pose, $\forall \beta\neq\alpha,\;W_{\beta}=V_{\beta}$, et $\forall l \notin \Gamma^{\alpha},\; g_{l}=f_{l}$. Ainsi on laisse inchangés tous les sommets différents de $\alpha$ et toutes les arêtes ne faisant pas intervenir $\alpha$. On considère les arêtes $l_{1},\dots,l_{n}$ qui vont en $\alpha$ et on note $h:\overset{n}{\underset{j=1}{\bigoplus}}V_{s(l_{j})}\rightarrow V_{\alpha}$ l'application qui à un vecteur $(v_{1},\dots,v_{n})$ associe la somme $f_{l_{1}}(v_{1})+\dots+f_{l_{n}}(v_{n})$. On pose alors $W_{\alpha}=\ker(h)$. On définie maintenant les applications $g_{l}$ pour $l\in\Gamma^{\alpha}$. Soit $i\in[\![1,n]\!]$, alors $g_{l_{i}}$ est la composition de l'injection de $\ker(h)$ dans $\overset{n}{\underset{j=1}{\bigoplus}}V_{s(l_{j})}$ et de la projection de $\overset{n}{\underset{j=1}{\bigoplus}}V_{s(l_{j})}$ dans $V_{s(l_{i})}$. 
\end{defi}
\begin{prop}
	L'application $F_{\alpha}^{+}:\mathscr{L}(\Gamma,\Lambda)\rightarrow\mathscr{L}(\Gamma,\sigma_{\alpha}\Lambda)$ est un foncteur.
\end{prop}

%Lignes à ajouter pour faire apparaître notre bibliographie
\clearpage
\bibliographystyle{unsrt}
\bibliography{biblio}

\end{document}

