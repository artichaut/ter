\documentclass[a4paper,10pt]{article}
\usepackage[utf8]{inputenc}
\usepackage[T1]{fontenc}
\usepackage[french]{babel}
\usepackage{amsmath,amssymb,amsthm}
\usepackage{mathrsfs}
\usepackage{graphicx}
\usepackage{tikz}
%\usepackage{listings}
%\usepackage{xcolor}

%opening
\title{Théorème de Gabriel}
\author{Maxime Buron, Édouard Rousseau\\
Travail encadré par Pierre-Guy Plamondon}

%Création des labels Théorème, Lemme, etc
\newtheorem{thm}{Théorème}[section]
\newtheorem{lm}{Lemme}[section]
\newtheorem{defi}{Définition}[section]
\newtheorem{prop}{Proposition}[section]
\newtheorem{ex}{Exemple}[section]

\begin{document}

\maketitle

%Le résumé de l'article
\begin{abstract}
	Ce mémoire a été écrit dans le cadre des Travaux Encadrés de Recherche (TER) au cours du M1 Mathématiques Fondamentales et Appliquées (MFA) de l'université Paris-Sud. Nous y démontrons un théorème dû à Pierre Gabriel datant de 1972, cependant, la preuve utilisée ici n'est pas la preuve orginale de Gabriel, nous suivons la preuve de Bernstein, Gel'Fand et Ponomarev issue de \cite{BGP72}.
\end{abstract}

\tableofcontents

\clearpage
        
\section{Notions préliminaires}
Afin de comprendre l'énoncé, puis la preuve, du théorème de Gabriel, nous avons besoin de notions. Dans toute cette partie, nous introduirons les concepts nécessaires à la bonne compréhension du lecteur.

\subsection{Carquois}
Le théorème de Gabriel donne de très beaux résultats sur les représentations de carquois. Cette notion est donc primordiale pour toute la suite de ce document.
\begin{defi}[Carquois]
	\label{carquois}
	Un carquois est un graphe orienté, où les boucles et les arêtes multiples sont autorisées.
\end{defi}
\begin{ex}
~
	\begin{center}
	\begin{tikzpicture}
		\draw[<->,>=latex] (0,0)--(0,1);
	\end{tikzpicture}
	\end{center}
\end{ex}

\begin{ex}
~
	\begin{center}
		\begin{tikzpicture}[scale=.4]
			\draw (-1,0) node[anchor=east]  {$A_n$};
			\foreach \x in {0,...,5}
			\draw[xshift=\x cm,thick] (\x cm,0) circle (.3cm);
			\draw[dotted,thick] (0.3 cm,0) -- +(1.4 cm,0);
			\foreach \y in {1.15,...,4.15}
			\draw[xshift=\y cm,thick] (\y cm,0) -- +(1.4 cm,0);
		\end{tikzpicture}
	\end{center}
\end{ex}

%Lignes à ajouter pour faire apparaître notre bibliographie
\clearpage
\bibliographystyle{unsrt}
\bibliography{biblio}
\end{document}

