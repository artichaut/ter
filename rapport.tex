\documentclass[a4paper,10pt]{article}
\usepackage[utf8]{inputenc}
\usepackage[T1]{fontenc}
\usepackage[french]{babel}
\usepackage{amsmath,amssymb,amsthm,amsopn}
\usepackage{mathrsfs}
\usepackage{graphicx}
\usepackage{tikz}
%\usepackage{listings}
%\usepackage{xcolor}

%opening
\title{Théorème de Gabriel}
\author{Maxime Buron, Édouard Rousseau\\
Travail encadré par Pierre-Guy Plamondon}

%Création des labels Théorème, Lemme, etc
\newtheorem{thm}{Théorème}[section]
\newtheorem{lm}{Lemme}[section]
\newtheorem{defi}{Définition}[section]
\newtheorem{prop}{Proposition}[section]
\newtheorem{ex}{Exemple}[section]

%Feuilles de styles Tikz
\tikzset{sommet/.style={circle, draw, fill,scale=0.4}}
\tikzset{fleche/.style={->,>=latex}}

%Raccourcis pour les opérateurs mathématiques (les espaces avant-après sont modifiés pour mieux rentrer dans les codes mathématiques usuels)
\DeclareMathOperator{\Ker}{Ker}
\DeclareMathOperator{\Id}{Id}
\DeclareMathOperator{\Img}{Im}
%Nouvelles commandes
\newcommand{\Fpm}{F_{\alpha}^{\pm}}
\newcommand{\Fp}{F_{\alpha}^{+}}
\newcommand{\Fm}{F_{\beta}^{-}}

%Fin des paramètres globaux et début du document.
\begin{document}

\maketitle

%Le résumé de l'article
\begin{abstract} 
	Ce mémoire a été écrit dans le cadre des Travaux Encadrés de Recherche (TER) au cours du M1 Mathématiques Fondamentales et Appliquées (MFA) de l'université Paris-Sud. Nous y démontrons un théorème dû à Pierre Gabriel datant de 1972, cependant, la preuve utilisée ici n'est pas la preuve orginale de Gabriel, nous suivons la preuve de Bernstein, Gel'Fand et Ponomarev issue de \cite{BGP72}.
\end{abstract}

\tableofcontents

\clearpage
        
\section{Notions préliminaires}
Afin de comprendre l'énoncé, puis la preuve, du théorème de Gabriel, nous avons besoin de notions. Dans toute cette partie, nous introduirons les concepts nécessaires à la bonne compréhension du lecteur.

\subsection{Carquois}
Le théorème de Gabriel donne de très beaux résultats sur les représentations de carquois. Cette notion est donc primordiale pour toute la suite de ce document.
\begin{defi}[Carquois]
	\label{carquois}
	Un carquois $(\Gamma,\Lambda)$ est constitué d'un graphe quelconque non orienté $\Gamma$ et d'une orientation $\Lambda$. On note  $\Gamma_0$ l'ensemble des sommets de $\Gamma$ et $\Gamma_1$ l'ensemble des arêtes non orientées de $\Gamma$. L'orientation $\Lambda$ est un couple de fonctions $(s,t)$ nommées recpectivement source et destination définies de $\Gamma_1$ dans $\Gamma_0$. 
\end{defi}
On considérera seulement des carquois connexes dans les exemples ci dessous.
	\begin{ex}
		\begin{tikzpicture}
			\node[sommet] (A) at (0,0) {};
			\node[sommet] (B) at (2,0) {};
			\draw[fleche] (A) to (B);
		\end{tikzpicture}
	\end{ex}
	\begin{ex}
		\begin{tikzpicture}
			\node[sommet] (A) at (0,0) {};
			\node[sommet] (B) at (2,0) {};
			\draw[fleche] (A) to[bend left] (B);
			\draw[fleche] (B) to[bend left] (A);
			\draw[fleche] (A.north) arc (0:348:.5);
		\end{tikzpicture}
	\end{ex}
	\begin{ex}
		\begin{tikzpicture}
			\node[sommet] (A) at (0,0) {};
			\node[sommet] (B) at (4,0) {};
			\node[sommet] (C) at (2,1) {};
			\node[sommet] (D) at (4,2) {};
			\node[sommet] (E) at (0,2) {};
			\draw[fleche] (A) to (C);
			\draw[fleche] (A) to (B);
			\draw[fleche] (E) to (A); 
			\draw[fleche] (C) to[bend left] (D);
			\draw[fleche] (C) to[bend right] (D);
			\draw[fleche] (E) to[bend left] (D);
			\draw[fleche] (B) to (C);
			\draw[fleche] (C) to (E);
			\draw[fleche] (B) to[bend left] (A);
			\draw[fleche] (E) to[bend left] (A);
			\draw[fleche] (A) to[bend left] (E);
		\end{tikzpicture}
	\end{ex}


\subsection{Représentations de carquois}
Maintenant que nous connaissons la définition d'un carquois, et que nous pouvons nous représenter l'objet, nous sommes capables d'aborder le coeur du sujet : les représentations de carquois. À l'issue de cette partie, nous serons même capables d'énoncer, et comprendre, une partie du théorème de Gabriel.

\begin{defi}[Représentation de carquois]
  Une représentation de carquois d'un carquois $(\Gamma,\Lambda)$ est un couple $(V,f)$ où $V$ est une liste indexée sur $\Gamma_0$ d'espaces vectoriels de dimension finie et $f$ est une liste indexée sur $\Gamma_1$ de morphismes d'espace vectoriel tels que pour tout $l \in \Gamma_1, f_l$ est définie de $V_{s(l)}$ dans $V_{t(l)}$.

\end{defi}

\subsection{Théorie des catégories}
Avant de traiter la preuve du thérorème de Gabriel, nous introduisons le langage de la théorie des catégories, celui nous aidera au court de la preuve. Les notations utilisées ici sont celle de \cite{A97}, et de plus amples détails sur les catégories peuvent y être trouvées, nous ne donnons ici qu'une brève présentation des notions qui nous serons utiles par la suite.

\subsection{Système de racines}
Afin de faire le lien entre les diagrammes de Dynkin et les représentations de carquois, il nous sera utile de savoir ce qu'est un système de racines, l'arrivée des diagrammes de Dynkin et des groupes de Weil nous permettra même de préciser le théorème de Gabriel que nous connaissions jusqu'à maintenant.



\clearpage
\section{Foncteurs images et foncteur de Coxeter}
Soit $(\Gamma,\Lambda)$ un carquois, on se place dans la catégorie $\mathscr{L}(\Gamma,\Lambda)$. Nous allons construire des foncteurs images, associés aux sommets $\alpha \in \Gamma_{0}$ de $\Gamma$ pour lesquels la direction des flèches le contenant est identique. On note $\Gamma^{\alpha}=\{ l\in \Gamma_{1} : s(l)=\alpha\text{ ou }t(l)=\alpha\}$ l'ensemble des arêtes contenant $\alpha$.

\begin{defi}
	On dit qu'un sommet $\alpha \in \Gamma_{0}$ est un puits, ou qu'il est (+)-accessible, si $\forall l \in \Gamma^{\alpha},\; t(l)=\alpha$, c'est-à-dire si toutes les arêtes contenant $\alpha$ arrivent en $\alpha$ et qu'il n'y a pas de boucles de $\alpha$ dans lui même.

On dit de même qu'un sommet $\beta \in \Gamma_{0}$ est une source, ou qu'il est (-)-accessible, si $\forall l \in \Gamma^{\beta},\; s(l)=\beta$, c'est-à-dire si toutes les arêtes contenant $\beta$ partent de $\beta$ et qu'il n'y a pas de boucles de $\beta$ dans lui même.
\end{defi}


Si $\gamma \in \Gamma_0$ est un sommet, on note $\sigma_{\gamma}\Lambda$ l'orientation obtenue à partir de $\Lambda$ en inversant le sens des flêches contenant $\gamma$.

Supposons que le sommet $\alpha \in \Gamma_{0}$ soit un puits avec l'orientation $\Lambda$.  Nous allons construire une application qui à un objet $(V,f)$ de $\mathscr{L}(\Gamma,\Lambda)$ associe un objet $(W,g)$ de $\mathscr{L}(\Gamma,\sigma_{\alpha}\Lambda)$. De même si $\beta \in \Gamma_0$ est une source, nous allons construire une autre application qui à un objet de $\mathscr L(\Gamma,\Lambda)$ associe un objet de $\mathscr L(\Gamma,\sigma_{\beta}\Lambda)$. 

\begin{defi}
	On notera l'application $F_{\alpha}^{+}$ et on aura $F_{\alpha}^{+}(V,f)=(W,g)$. On pose, $\forall \gamma\neq\alpha,\;W_{\gamma}=V_{\gamma}$, et $\forall l \notin \Gamma^{\alpha},\; g_{l}=f_{l}$. Ainsi on laisse inchangés tous les sommets différents de $\alpha$ et toutes les arêtes ne faisant pas intervenir $\alpha$. On considère les arêtes $l_{1},\dots,l_{n}$ qui vont en $\alpha$ et on note $h:\overset{n}{\underset{j=1}{\bigoplus}}V_{s(l_{j})}\rightarrow V_{\alpha}$ l'application qui à un vecteur $(v_{1},\dots,v_{n})$ associe la somme $f_{l_{1}}(v_{1})+\dots+f_{l_{n}}(v_{n})$. On pose alors $W_{\alpha}=\Ker(h)$. On définie maintenant les applications $g_{l}$ pour $l\in\Gamma^{\alpha}$. Soit $i\in[\![1,n]\!]$, alors $g_{l_{i}}$ est la composition de l'injection de $\Ker(h)$ dans $\overset{n}{\underset{j=1}{\bigoplus}}V_{s(l_{j})}$ et de la projection de $\overset{n}{\underset{j=1}{\bigoplus}}V_{s(l_{j})}$ dans $V_{s(l_{i})}$. 

	On notera de même, $F^{-}_{\beta}$, l'application et on aura $F^{-}_{\beta}(V,f) = (W,g)$. On pose, $\forall \gamma \neq \alpha, W_\gamma = V_\gamma$, et $\forall l \notin \Gamma_\beta, f_l = g_l$. En considérant les arêtes $l_1, \dots, l_n$ qui partent de $\beta$, on note $\tilde{h} : V_\beta \rightarrow \overset{n}{\underset{j=1}{\bigoplus}}V_{t(l_{j})}$ et qui à un vecteur $v \in V_\beta$ associe le vecteur $(f_1(v),\dots, f_n(v))$. On pose alors $W_\beta = \overset{n}{\underset{j=1}{\bigoplus}}V_{t(l_{j})}/\Img(\tilde{h})$. On pose pour $i\in [\![1,n]\!]$, alors $g_{l_i}$ comme la composition de l'inclusion naturelle de $V_{l_i}$ dans $\overset{n}{\underset{j=1}{\bigoplus}}V_{t(l_{j})}$ et de la projection dans $\overset{n}{\underset{j=1}{\bigoplus}}V_{t(l_{j})}/\Img(\tilde h)$.

\end{defi}

\begin{prop}
	L'application $F_{\alpha}^{+}:\mathscr{L}(\Gamma,\Lambda)\rightarrow\mathscr{L}(\Gamma,\sigma_{\alpha}\Lambda)$ est un foncteur. De même l'application $F_{\beta}^{-}:\mathscr{L}(\Gamma,\Lambda)\rightarrow\mathscr{L}(\Gamma,\sigma_{\beta}\Lambda)$ est un foncteur.
\end{prop}

\begin{proof}
	On se place dans le même cadre que lors des définitions, ainsi on a $(V,f)\in\mathscr{L}(\Gamma,\Lambda)$ une représentation du carquois $\Gamma$. On note toujours $F_{\alpha}^{+}(V,f)=(W,g)\in\mathscr{L}(\Gamma,\sigma_{\alpha}\Lambda)$ la représentation image de $(V,f)$ par l'application $F_{\alpha}^{+}$. On a déjà la définition de $F_{\alpha}^{+}$ sur les objets des catégories $\mathscr{L}(\Gamma,\Lambda)$ et $\mathscr{L}(\Gamma,\sigma_{\alpha}\Lambda)$, et pour tout $(\beta,l)\in\Gamma_{0}\times\Gamma_{1}$, on a bien que $W_{\beta}$ est un espace vectoriel, et que $g_{l}$ est une application linéaire. Il nous reste donc à définir $F_{\alpha}^{+}$ sur les morphismes de représentations pour obtenir son caractère fonctoriel. Ainsi, soient $(V,f)$ et $(V',f')$ deux représentations de $\Gamma$, on note encore $l_{1},\dots,l_{n}\in\Gamma^{\alpha}$ les arêtes en direction du puits $\alpha$, et soit $\varphi=(\varphi_{\beta})_{\beta\in\Gamma_{0}}$ un morphisme de représentation entre $(V,f)$ et $(V',f')$. On note $(W,g)$ et $(W',g')$ les représentations images par $F_{\alpha}^{+}$ de $(V,f)$ et $(V',f')$. On note également $\gamma=(\gamma_{\beta})_{\beta\in\Gamma_{0}}$ le morphisme image de $\varphi$ par $F_{\alpha}^{+}$, qu'il nous faut encore définir. On pose donc :
	\[	
		\gamma_{\alpha}(v_{1},\dots,v_{n})=(\varphi_{s(l_{1})}(v_{1}),\dots,\varphi_{s(l_{n})}(v_{n}))
	\]
	\[
		\forall \beta\neq\alpha,\; \gamma_{\beta}=\varphi_{\beta}.
	\]
	Assurons nous tout d'abord du caractère bien défini de $\gamma$. Pour tout $\beta\neq\alpha$, on a $V_{\beta}=W_{\beta}$, $V'_{\beta}=W'_{\beta}$ et $\gamma_{\beta}=\varphi_{\beta}$, le morphisme $\gamma$ est donc bien défini pour tout sommet $\beta\neq\alpha$. Il nous reste à vérifier que $\gamma_{\alpha}$ est bien défini. Soit $(v_{1},\dots,v_{n})\in\Ker(h)$, il nous faut voir que $\gamma_{\alpha}(v_{1},\dots,v_{n})\in\Ker(k)$, où
	\[
\begin{array}{lccc}
	h : & \overset{n}{\underset{j=1}{\bigoplus}}V_{s(l_{j})}&\rightarrow & V_{\alpha} \\ 
	& (v_{1},\dots,v_{n})&\mapsto & \underset{i=1}{\overset{n}{\sum}}f_{l_{i}}(v_{i})\\
\end{array}
	\]	
	\[
\begin{array}{lccc}
	k : & \overset{n}{\underset{j=1}{\bigoplus}}V'_{s(l_{j})}&\rightarrow & V'_{\alpha} \\ 
	& (v'_{1},\dots,v'_{n})&\mapsto & \underset{i=1}{\overset{n}{\sum}}f'_{l_{i}}(v'_{i}).\\
\end{array}
	\]
Ainsi :
\[
\begin{array}{rl}
	k(\gamma_{\alpha}(v_{1},\dots,v_{n})) & = \underset{i=1}{\overset{n}{\sum}}f'_{l_{i}}(\varphi_{s(l_{i})}(v_{i})) \\ 
	& = \underset{i=1}{\overset{n}{\sum}} \varphi_{\alpha}(f_{l_{i}}(v_{i}))\\
	& = \varphi_{\alpha}(\underset{i=1}{\overset{n}{\sum}}f_{l_{i}}(v_{i}))\\
	& = \varphi_{\alpha}(0) \\
	& = 0 \\
\end{array}
\]
donc $\gamma_{\alpha}$ est bien définie. Voyons désormais que $\gamma$ est bien un morphisme de représentations entre $(W,g)$ et $(W',g')$. Comme pour tout $\beta\neq\alpha$, on a $\gamma_{\beta}=\varphi_{\beta}$, et pour tout $l\notin\Gamma^{\alpha}$, on a $g_{l}=f_{l}$ et $g'_{l}=f'_{l}$, il vient facilement que $g'_{l_{i}}\gamma_{s(l_{i})}=\gamma_{t(l_{i})}g_{l_{i}}$ dès lors que $l\notin\Gamma^{\alpha}$. Regardons maintenant le cas où $l\in\Gamma^{\alpha}$, il existe alors $i\in[\![1,n]\!]$ tel que $l=l_{i}$, et on a $t(l_{i})=\alpha$. Soit $(v_{1},\dots,v_{n})\in W_{\alpha}$, il vient, d'une part :

\[
\begin{array}{ll}
	g'_{l_{i}}\gamma_{\alpha}(v_{1},\dots,v_{n})&=g'_{l_{i}}(\varphi_{s(l_{1})}(v_{1}),\dots,\varphi_{s(l_{n})}(v_{n}))\\
	&=\varphi_{s(l_{i})}(v_{i})
\end{array}
\]
et d'autre part :
\[
\begin{array}{ll}
	\gamma_{s(l_{i})}g_{l_{i}}(v_{1},\dots,v_{n})&=\gamma_{s(l_{i})}(v_{i})\\
	&=\varphi_{s(l_{i})}(v_{i}).
\end{array}
\]
Ainsi, on a bien que $\forall l \in\Gamma_{1},\;g'_{l}\gamma_{t(l)}=\gamma_{s(l)}g_{l}$, ce qui prouve que $\gamma$ est un morphisme de représentations de $(W,g)$ vers $(W',g')$. Il nous reste donc à voir que $F_{\alpha}^{+}$ respecte la composition des morphismes et qu'il envoie l'identité sur l'identité. À cet effet, posons $\Id = (\Id_{V_{\beta}})_{\beta\in\Gamma_{0}}$ le morphisme idendité de $(V,f)$ dans $(V,f)$, c'est à dire que pour tout sommet $\beta\in\Gamma_{0}$, $\Id_{V_{\beta}}$ est l'application identité de $V_{\beta}$ dans lui-même. On a :
\[
	\forall \beta\neq\alpha,\;F_{\alpha}^{+}(\Id)_{\beta}=\Id_{V_{\beta}}
\]
\[
	F_{\alpha}^{+}(\Id)=(\Id_{V_{s(l_{1})}},\dots,\Id_{V_{s(l_{n})}})=\Id_{\Ker(h)}.
\]
Le morphisme identité de $(V,f)$ est donc bien envoyé sur le morphisme identité de $F_{\alpha}^{+}(V,f)$. Posons désormais $(V,f)$,$(V',f')$ et $(V'',f'')$ trois représentations de $(\Gamma,\Lambda)$, $\varphi$ (respectivement $\varphi '$) un morphisme de représentation de $(V,f)$ dans $(V',f')$ (respectivement de $(V',f')$ dans $(V'',f'')$), ainsi que $(W,g)$, $(W',g')$, $(W'',g'')$, $\gamma$ et $\gamma '$ leurs images par $F_{\alpha}^{+}$. Il vient :
\[
	\forall \beta\neq\alpha,\;F_{\alpha}^{+}(\varphi')_{\beta}F_{\alpha}^{+}(\varphi)_{\beta}=\varphi_{\beta}'\varphi_{\beta}=F_{\alpha}^{+}(\varphi'\varphi)_{\beta}
\]
\[
	\begin{array}{ll}
		\forall (v_{1},\dots,v_{n})\in\Ker(h),& \\
		F_{\alpha}^{+}(\varphi')_{\alpha}F_{\alpha}^{+}(\varphi)_{\alpha}(v_{1},\dots,v_{n})&=F_{\alpha}^{+}(\varphi')_{\alpha}(\varphi_{s(l_{1})}(v_{1}),\dots,\varphi_{s(l_{n})}(v_{n}))\\
		&=(\varphi'_{s(l_{1})}(\varphi_{s(l_{1})}(v_{1})),\dots,\varphi'_{s(l_{n})}(\varphi_{s(l_{n})}(v_{n})))\\
		&=F_{\alpha}^{+}(\varphi'\varphi)_{\alpha}(v_{1},\dots,v_{n})

	\end{array}
\]
Ainsi la composition de deux morphismes est bien respectée par $F_{\alpha}^{+}$, et cela conclue la preuve pour cette application.

On a déjà définie l'action de $F_{\beta}^{-}$ sur les représentations de la catégorie $\mathscr L(\Gamma,\Lambda)$ à valeurs dans la catégorie $\mathscr L(\Gamma,\sigma_{\beta}\Lambda)$. Il s'agit maintenant de définir l'action de $F_{\beta}^{-}$ sur les morphismes de représentations. Soit $(V,f)$ et $(V',f')$ deux représentations et $(\varphi_{\gamma})_{\gamma \in \Gamma_{0}}$ un morphisme entre ces deux représentations dans $\mathscr L(\Gamma,\Lambda)$. Posons pout tout $\gamma \neq \beta, F_{\beta}^{-}(\varphi_{\gamma}) = \varphi_{\gamma}$, car $F_{\beta}^{-}$ laisse inchangé les espaces vectoriels associés aux sommets différents de $\beta$.
\end{proof}
\begin{defi}
	Soit $\beta\in\Gamma_{0}$ une source suivant l'orientation $\Lambda$, alors $\beta$ est un puits suivant l'orientation $\sigma_{\beta}\Lambda$. On peut donc définir le foncteur : $F_{\beta}^{-}F_{\beta}^{+}:\mathscr{L}(\Gamma,\Lambda)\rightarrow\mathscr{L}(\Gamma,\Lambda).$ Soit $(V,f)\in\mathscr{L}(\Gamma,\Lambda)$, on note $(Z,h)=F_{\beta}^{-}F_{\beta}^{+}(V,f)$ l'image de $(V,f)$ par $F_{\beta}^{-}F_{\beta}^{+}$ et on construit alors une application $i_{V}^{\beta}:(Z,h)\rightarrow (V,f)$ de la manière suivante :
	\[
		\forall \gamma\neq\beta,\;Z_{\gamma}=V_{\gamma}\text{, on pose donc : }(i_{V}^{\beta})_{\gamma}=\Id_{V_{\gamma}}.
	\]
	Pour $(i_{V}^{\beta})_{\beta}$, on prend l'application naturelle : 
	\[
		Z_{\beta}=\underset{l\in\Gamma^{\beta}}{\oplus} V_{s(l)}/\Img \tilde{h}=\underset{l\in\Gamma^{\beta}}{\oplus} V_{s(l)}/\Ker h \rightarrow V.
	\]
\end{defi}
\begin{prop}
	L'application $i_{V}^{\beta}$ est un morphisme de $(Z,h)$ dans $(V,f)$.
\end{prop}
\begin{proof}
	Notons $(W,g)=F_{\beta}^{+}(V,f)$ l'image de $(V,f)$ par $F_{\beta}^{+}$. Dans la définition de $(i_{V}^{\beta})_{\beta}$, on a implicitement déclaré que $\Ker h=\Img\tilde h$, nous allons le prouver. Considérons la suite d'applications suivante :
	\[
		W_{\beta}\overset{\tilde h}{\longrightarrow}\underset{l\in\Gamma^{\beta}}{\oplus}V_{s(l)}\overset{h}{\longrightarrow}V_{\beta},
	\]
on a alors :
\[
	\begin{array}{lll}
		\Img \tilde h &=& \left\{ (g_{l_{1}}(v_{1}),\dots,g_{l_{n}}(v_{n})) \;|\; (v_{1},\dots,v_{n})\in W_{\beta} \right\}\\
		&=& \left\{ (v_{1},\dots,v_{n})\;|\;(v_{1},\dots,v_{n})\in\Ker h \right\}\\
		&=& \Ker h.
\end{array}
\]
Car $W_{\beta}=\Ker h$ par définition et pour tout $i\in[\![1,n]\!]$, $g_{l_{i}}$ est la composée de l'injection $\Ker h \hookrightarrow \oplus_{l\in\Gamma^{\beta}}V_{s(l)}$ avec la projection $\oplus_{l\in\Gamma^{\beta}}V_{s(l)}\twoheadrightarrow V_{s(l_{i})}$.
\end{proof}

\begin{lm}
	On a les résultats suivants :
	\begin{enumerate}
		\item $F_{\alpha}^{\pm}(V_{1}\oplus V_{2})=F_{\alpha}^{\pm}(V_{1})\oplus F_{\alpha}^{\pm}(V_{2})$ ;
		\item $p_{V}^{\alpha}$ est surjectif et $i_{V}^{\beta}$ est injectif ;
		\item dimensions ;
		\item $\forall \gamma\neq\alpha,\;(\Ker p_{V}^{\alpha})_{\gamma}=\left\{ 0 \right\}$ et $\forall \gamma\neq\beta,\;(V/\Img i_{V}^{\beta})_{\gamma}=\left\{ 0 \right\}$ ;
		\item si $V$ est de la forme $F_{\alpha}^{+}(W)$ (respectivement $F_{\beta}^{-}(W)$), alors $p_{V}^{\alpha}$ ($i_{V}^{\beta}$) est un isomorphisme ;
		\item $V$ est isomorphe à $F_{\alpha}^{+}F_{\alpha}^{-}(V)\oplus V/\Img i_{V}^{\beta}$, de même $V$ est isomorphe à $F_{\alpha}^{+}F_{\alpha}^{-}(V)\oplus \Ker p_{V}^{\alpha}$.
	\end{enumerate}
\end{lm}

%Lignes à ajouter pour faire apparaître notre bibliographie
\clearpage
\bibliographystyle{unsrt}
\bibliography{biblio}

\end{document}


%%% Local Variables:
%%% mode: latex
%%% TeX-master: t
%%% End:
