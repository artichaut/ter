\documentclass[a4paper,10pt]{article}
\usepackage[utf8]{inputenc}
\usepackage[T1]{fontenc}
\usepackage[french]{babel}
\usepackage{amsmath,amssymb,amsthm,amsopn}
\usepackage{mathrsfs}
\usepackage{graphicx}
\usepackage{tikz}
\usepackage{array}
%\usepackage{listings}
%\usepackage{xcolor}

%opening
\title{Théorème de Gabriel}
\author{Maxime Buron, Édouard Rousseau\\
Travail encadré par Pierre-Guy Plamondon}

%Création des labels Théorème, Lemme, etc
\newtheorem{thm}{Théorème}[section]
\newtheorem{lm}[thm]{Lemme}%[section]
\newtheorem{defi}[thm]{Définition}%[section]
\newtheorem{prop}[thm]{Proposition}%[section]
\newtheorem{ex}[thm]{Exemple}%[section]
\newtheorem{cor}[thm]{Corollaire}%[section]

%Feuilles de styles Tikz
\usetikzlibrary{matrix}
\usetikzlibrary{calc}
\tikzset{sommet/.style={circle, draw, fill,scale=0.4}}
\tikzset{fleche/.style={->,>=latex}}
\tikzset{trait/.style={thick}}

%Raccourcis pour les opérateurs mathématiques (les espaces avant-après sont modifiés pour mieux rentrer dans les codes mathématiques usuels)
\DeclareMathOperator{\Ker}{Ker}
\DeclareMathOperator{\Id}{Id}
\DeclareMathOperator{\Img}{Im}
\DeclareMathOperator{\Card}{Card}
\DeclareMathOperator{\Vect}{Vect}
%Nouvelles commandes
\newcommand{\F}[3]{F_{#1}^{#2}(#3)}
\newcommand{\ps}[2]{\left\langle#1,#2\right\rangle}
\newcommand{\EG}{\mathscr{E}_\Gamma}
\newcommand{\ent}[2]{[\![#1,#2]\!]}
\newcommand{\dps}{\displaystyle}

%Fin des paramètres globaux et début du document.
\begin{document}

\maketitle

%Le résumé de l'article
\begin{abstract} 
	Ce mémoire a été écrit dans le cadre des Travaux Encadrés de Recherche (TER) au cours du M1 Mathématiques Fondamentales et Appliquées (MFA) de l'université Paris-Sud. Nous y démontrons un théorème dû à Pierre Gabriel datant de 1972, cependant, la preuve utilisée ici n'est pas la preuve orginale de Gabriel, nous suivons la preuve de Bernstein, Gel'Fand et Ponomarev issue de \cite{BGP72}. On introduit tout d'abord des notions fondamentales à la compréhension de cette preuve, comme les représentations de carquois, la thérorie des catégories, et les systèmes de racines. Puis on met en place des outils, tels que foncteur de Coxeter. Ces outils nous permettront, finalement, de prouver le théorème de Gabriel.
\end{abstract}

\tableofcontents

\clearpage
\section*{Quelques notations}
Rassemblons ici quelques notations que nous utiliserons sans rappel dans la suite du document. Nous noterons $fg$ plutôt que $f\circ g$, la composition entre les fonctions $f$ et $g$, $fg$ ne sera jamais un produit point par point. Si $f$ est une application, on note $\Img f$ son image et $\Ker f$ son noyau. On utilise la flèche $\hookrightarrow$ pour désigner une injection, et la flèche $\twoheadrightarrow$ pour désigner une surjection. Nous utiliserons l'abus de notation $x=0$ où $x$ peut représenter des objets de natures diverses, $0$ désignera alors l'élément nul correspondant à la nature de l'objet $x$. Lorsqu'il n'y a pas d'ambigüité, ou par souci de simplification, nous noterons $V$ plutôt que $(V,f)$ pour une représentation de carquois. Les espaces vectoriels utilisés sont tous des $\mathbb K$-espaces vectoriels, où $\mathbb K$ est un corps fixé. Si $V$ et $V'$ sont deux espaces vectoriels, on note $V\oplus V'$ leur somme directe. Si $f$ et $g$ sont deux applications linéaires, on note $f\oplus g$ leur somme directe. Si $A$ et $B$ sont deux objets, on note $A/B$ leur quotient, la définition de quotient varie alors en fonction de la nature de $A$ et $B$. Si $X$ est un ensemble, on note $\Id_X$ l'application identité de $X$ dans lui-même.  On note $[\![1,n]\!]$ l'ensemble des entiers naturels compris entre $1$ et $n$. 
        
\section{Notions préliminaires}
Afin de comprendre l'énoncé, puis la preuve, du théorème de Gabriel, nous avons besoin de notions. Dans toute cette partie, nous introduirons les concepts nécessaires à la bonne compréhension du lecteur.

\subsection{Carquois}
Le théorème de Gabriel donne de très beaux résultats sur les représentations de carquois. Cette notion est donc primordiale pour toute la suite de ce document.
\begin{defi}[Carquois]
	\label{carquois}
	Un \emph{carquois} $(\Gamma,\Lambda)$ est constitué d'un graphe quelconque non orienté $\Gamma$ et d'une orientation $\Lambda$. On note  $\Gamma_0$ l'ensemble des sommets de $\Gamma$ et $\Gamma_1$ l'ensemble des arêtes non orientées de $\Gamma$. L'orientation $\Lambda$ est un couple de fonctions $(s,t)$ nommées recpectivement source et destination définies de $\Gamma_1$ dans $\Gamma_0$. 
\end{defi}
On considérera seulement des carquois connexes dans les exemples ci dessous.
	\begin{ex}
		\begin{tikzpicture}
			\node[sommet] (A) at (0,0) {};
			\node[sommet] (B) at (2,0) {};
			\draw[fleche] (A) to (B);
		\end{tikzpicture}
	\end{ex}
	\begin{ex}
		\begin{tikzpicture}
			\node[sommet] (A) at (0,0) {};
			\node[sommet] (B) at (2,0) {};
			\draw[fleche] (A) to[bend left] (B);
			\draw[fleche] (B) to[bend left] (A);
			\draw[fleche] (A.north) arc (0:348:.5);
		\end{tikzpicture}
	\end{ex}
	\begin{ex}
		\begin{tikzpicture}
			\node[sommet] (A) at (0,0) {};
			\node[sommet] (B) at (4,0) {};
			\node[sommet] (C) at (2,1) {};
			\node[sommet] (D) at (4,2) {};
			\node[sommet] (E) at (0,2) {};
			\draw[fleche] (A) to (C);
			\draw[fleche] (A) to (B);
			\draw[fleche] (E) to (A); 
			\draw[fleche] (C) to[bend left] (D);
			\draw[fleche] (C) to[bend right] (D);
			\draw[fleche] (E) to[bend left] (D);
			\draw[fleche] (B) to (C);
			\draw[fleche] (C) to (E);
			\draw[fleche] (B) to[bend left] (A);
			\draw[fleche] (E) to[bend left] (A);
			\draw[fleche] (A) to[bend left] (E);
		\end{tikzpicture}
	\end{ex}


\subsection{Représentations de carquois}
Maintenant que nous connaissons la définition d'un carquois, et que nous pouvons nous représenter l'objet, nous sommes capables d'aborder le coeur du sujet : les représentations de carquois. À l'issue de cette partie, nous serons même capables d'énoncer, et comprendre, une partie du théorème de Gabriel.
\begin{defi}[Représentation de carquois]
	Une \emph{représentation de carquois} d'un carquois $(\Gamma,\Lambda)$ est un couple $(V,f)$ où $V$ est une liste indexée sur $\Gamma_0$ d'espaces vectoriels de dimension finie et $f$ est une liste indexée sur $\Gamma_1$ de morphismes d'espace vectoriel tels que pour tout $l \in \Gamma_1, f_l$ est définie de $V_{s(l)}$ dans $V_{t(l)}$.
\end{defi}
\begin{ex}[Une représentation]
\end{ex}
\begin{ex}[Une deuxième]
\end{ex}
\begin{defi}[Représentation nulle]
	Soit $(\Gamma,\Lambda)$ un carquois, la \emph{représentation nulle} de $(\Gamma,\Lambda)$ est la représentation $(V,f)$ où $\forall\gamma\in\Gamma^{0},\;V_{\gamma}=\left\{ 0 \right\}$ et où $\forall l \in\Gamma_{1},\;f_{l}$ est l'application linéaire nulle. On note $(V,f)=0$.
\end{defi}
\begin{defi}[Somme directe de deux représentations de carquois]
	Soit $(\Gamma,\Lambda)$ un carquois, $(V,f)$ et $(V',f')$ deux représentations de ce carquois. La \emph{somme directe} de $(V,f)$ et $(V',f')$ est la représentation $(W,g)$ définie par $\forall\gamma\in\Gamma_{0},\;W_{\gamma}=V_{\gamma}\oplus V_{\gamma}'$ et $\forall l \in\Gamma_{1},\;g_{l}=f_{l}\oplus f_{l}'$. On note $(W,g)=(V,f)\oplus (V',f')$.
\end{defi}
\begin{defi}[Représentation indécomposable]
	Soit $(\Gamma,\Lambda)$ un carquois et $(V,f)$ une représentation non nulle de ce carquois. On dit que $(V,f)$ est une \emph{représentation indécomposable} si on ne peut pas l'écrire comme somme de deux représentations non nulles. De manière équivalente, on dit que $(V,f)$ est une représentation indécomposable si pour toutes représentations $(V^{1},f^{1})$, $(V^{2},f^{2})$, $(V,f)=(V^{1},f^{1})\oplus(V^{2},f^{2})$ implique $(V^{1},f^{1})=0$ ou $(V^{2},f^{2})=0$.
\end{defi}
\begin{ex}[Représentations irréductibles associées aux sommets]
\label{irreductible}
Soit $(\Gamma,\Lambda)$ un carquois. On appelle représentations irréductibles, et on note $L_\gamma$, où $\gamma\in\Gamma_{0}$, les représentations telles que $\forall\delta\neq\gamma,\;(L_\gamma)_\delta=\left\{ 0 \right\}$,  $(L_\gamma)_\gamma=\mathbb K$ et $\forall l \in\Gamma_1,\;f_l=0$. $L_\gamma$ est alors une représentation indécomposable.
\end{ex}
\begin{ex}[Une représentation indécomposable plus grosse]
\end{ex}
\begin{defi}[Morphisme de représentation]
	Soit $(\Gamma,\Lambda)$ un carquois et $(V,f)$,$(W,g)$ deux représentations de ce carquois. On dit que $\varphi=(\varphi_{\gamma})_{\gamma\in\Gamma_{0}}$ est un \emph{morphisme de représentations} de $(V,f)$ dans $(W,g)$ si, pour tout $l\in\Gamma_{1}$, le diagramme suivant commute :
	\[
	\begin{tikzpicture}
	\matrix (m) [matrix of math nodes,row sep=3em,column sep=4em,minimum width=2em]
	  {
		  V_{s(l)} & V_{t(l)} \\
		  W_{s(l)} & W_{t(l)} \\};
	\path[-stealth]
	(m-1-1) edge node [left] {$\varphi_{s(l)}$} (m-2-1)
		edge node [above] {$f_{l}$} (m-1-2)
	(m-2-1) edge node [below] {$g_{l}$} (m-2-2)
	(m-1-2) edge node [right] {$\varphi_{t(l)}$} (m-2-2);
	\end{tikzpicture}
\]
c'est-à-dire si $\forall l \in\Gamma_{1},\;\varphi_{t(l)}f_{l}=g_{l}\varphi_{s(l)}$.
\end{defi}
\begin{ex}[Morphisme de représentation]
\end{ex}
\begin{defi}[Représentation quotient]
\end{defi}
\begin{defi}[Sous-représentations]
\end{defi}
\subsection{Théorie des catégories}
Avant de traiter la preuve du thérorème de Gabriel, nous introduisons le langage de la théorie des catégories, qui nous aidera au court de la preuve. Les notations utilisées ici sont celle de \cite{A97}, et de plus amples détails sur les catégories peuvent y être trouvées, nous ne donnons ici qu'une brève présentation des notions qui nous serons utiles par la suite.
\begin{defi}[Catégorie]
\end{defi}
\begin{defi}[Foncteur]
\end{defi}
\subsection{Diagrammes de Dynkin}
Nous faisons ici un inventaire de certaines catégories de graphes non orientés et sans boucles, qui joueront un rôle fondamental dans le théorème de Gabriel. L'indice de ces graphes correspond au nombre de sommets qu'ils contiennent. 
\begin{center}
	\begin{tikzpicture}
    			\foreach \x in {0,...,5}
			\node[sommet] () at (\x,0) {};
    			\foreach \x in {0,...,1}
			\draw[thick] (\x,0) --+(1,0);
    			\foreach \x in {3,...,4}
			\draw[thick] (\x,0) --+(1,0);
			\draw[dotted, thick] (2,0) --+(1,0);
			\node () at (-1,0) {$A_n$};%Fin dessin An
    			\foreach \x in {0,...,5}
			\node[sommet] () at (\x,-1) {};
    			\foreach \y in {0,...,1}
			\draw[thick] (\y,-1) -- +(1cm,0);
			\foreach \y in {3,...,4}
			\draw[thick] (\y,-1) -- +(1cm,0);
			\draw[dotted, thick] (2,-1) -- (3,-1);
			\foreach \y in {-0.5,.5}
			\node[sommet] () at (5.5,\y-1) {};
			\foreach \y in {-1.5,-0.5}
			\draw[thick] (5.5,\y) to (5,-1);
			\node () at (-1,-1) {$D_n$};%Fin dessin Dn
			\foreach \x in {0,...,4}
			\node[sommet] () at (\x,-2) {};
			\foreach \x in {0,...,3}
			\draw[thick] (\x,-2) --+(1,0);
			\node[sommet] () at (2,-1.5) {};
			\draw[thick] (2,-2)--+(0,.5);
			\node () at (-1,-2) {$E_6$};%Fin E_6
			\foreach \x in {0,...,5}
			\node[sommet] () at (\x,-3) {};
			\foreach \x in {0,...,4}
			\draw[thick] (\x,-3) --+(1,0);
			\node[sommet] () at (2,-2.5) {};
			\draw[thick] (2,-3)--+(0,.5);
			\node () at (-1,-3) {$E_7$};%Fin E_7
			\foreach \x in {0,...,6}
			\node[sommet] () at (\x,-4) {};
			\foreach \x in {0,...,5}
			\draw[thick] (\x,-4) --+(1,0);
			\node[sommet] () at (2,-3.5) {};
			\draw[thick] (2,-4)--+(0,.5);
			\node () at (-1,-4) {$E_8$};%Fin E_8
	\end{tikzpicture}
\end{center}
On appelle ces graphes les diagrammes de Dynkin de type $A_n,D_n,E_6,E_7$ et $E_8$, ou encore diagrammes de Dynkin simplement lacés.
\subsection{Système de racines}
Afin de faire le lien entre les diagrammes de Dynkin simplement lacés et les représentations de carquois, il nous sera utile de savoir ce qu'est un système de racines, l'arrivée des diagrammes de Dynkin et des groupes de Weyl nous permettra de mieux comprendre les outils utilisés lors de la démonstration du théorème de Gabriel.

\clearpage
\section{Foncteurs de réflexion et foncteurs de Coxeter}
Soit $(\Gamma,\Lambda)$ un carquois, on se place dans la catégorie $\mathscr{L}(\Gamma,\Lambda)$. Nous allons construire des foncteurs images, associés aux sommets $\alpha \in \Gamma_{0}$ de $\Gamma$ pour lesquels la direction des flèches le contenant est identique. On note $\Gamma^{\alpha}=\{ l\in \Gamma_{1} : s(l)=\alpha\text{ ou }t(l)=\alpha\}$ l'ensemble des arêtes contenant $\alpha$.

\begin{defi}
	On dit qu'un sommet $\beta \in \Gamma_{0}$ est un \emph{puits}, ou qu'il est (+)-accessible, si $\forall l \in \Gamma^{\beta},\; t(l)=\beta$, c'est-à-dire si toutes les arêtes contenant $\beta$ arrivent en $\beta$ et qu'il n'y a pas de boucles de $\beta$ dans lui même.

	On dit de même qu'un sommet $\alpha \in \Gamma_{0}$ est une \emph{source}, ou qu'il est (-)-accessible, si $\forall l \in \Gamma^{\alpha},\; s(l)=\alpha$, c'est-à-dire si toutes les arêtes contenant $\alpha$ partent de $\alpha$ et qu'il n'y a pas de boucles de $\alpha$ dans lui même.
\end{defi}


Si $\gamma \in \Gamma_0$ est un sommet, on note $\sigma_{\gamma}\Lambda$ l'orientation obtenue à partir de $\Lambda$ en inversant le sens des flèches contenant $\gamma$.

Supposons que le sommet $\alpha \in \Gamma_{0}$ soit un puits avec l'orientation $\Lambda$.  Nous allons construire une application qui à un objet $(V,f)$ de $\mathscr{L}(\Gamma,\Lambda)$ associe un objet $(W,g)$ de $\mathscr{L}(\Gamma,\sigma_{\alpha}\Lambda)$. De même si $\beta \in \Gamma_0$ est une source, nous allons construire une autre application qui à un objet de $\mathscr L(\Gamma,\Lambda)$ associe un objet de $\mathscr L(\Gamma,\sigma_{\beta}\Lambda)$. 

\begin{defi}
	On notera l'application $F_{\beta}^{+}$ et on aura $F_{\beta}^{+}(V,f)=(W,g)$. On pose, $\forall \gamma\neq\beta,\;W_{\gamma}=V_{\gamma}$, et $\forall l \notin \Gamma^{\beta},\; g_{l}=f_{l}$. Ainsi on laisse inchangés tous les sommets différents de $\beta$ et toutes les arêtes ne faisant pas intervenir $\beta$. On considère les arêtes $l_{1},\dots,l_{n}$ qui vont en $\beta$ et on note $h:\overset{n}{\underset{j=1}{\bigoplus}}V_{s(l_{j})}\rightarrow V_{\beta}$ l'application qui à un vecteur $(v_{1},\dots,v_{n})$ associe la somme $f_{l_{1}}(v_{1})+\dots+f_{l_{n}}(v_{n})$. On pose alors $W_{\beta}=\Ker(h)$. On définit maintenant les applications $g_{l}$ pour $l\in\Gamma^{\beta}$. Soit $i\in[\![1,n]\!]$, alors $g_{l_{i}}$ est la composition de l'injection de $\Ker(h)$ dans $\overset{n}{\underset{j=1}{\bigoplus}}V_{s(l_{j})}$ et de la projection de $\overset{n}{\underset{j=1}{\bigoplus}}V_{s(l_{j})}$ dans $V_{s(l_{i})}$. 

	On notera de même, $F^{-}_{\alpha}$, l'application et on aura $F^{-}_{\alpha}(V,f) = (W,g)$. On pose, $\forall \gamma \neq \alpha, W_\gamma = V_\gamma$, et $\forall l \notin \Gamma_\alpha, f_l = g_l$. En considérant les arêtes $l_1, \dots, l_n$ qui partent de $\alpha$, on note $\widetilde{h} : V_\alpha \rightarrow \overset{n}{\underset{j=1}{\bigoplus}}V_{t(l_{j})}$ et qui à un vecteur $v \in V_\alpha$ associe le vecteur $(f_1(v),\dots, f_n(v))$. On pose alors $W_\alpha = \overset{n}{\underset{j=1}{\bigoplus}}V_{t(l_{j})}/\Img(\widetilde{h})$. On pose pour $i\in [\![1,n]\!]$, alors $g_{l_i}$ comme la composition de l'inclusion naturelle de $V_{l_i}$ dans $\overset{n}{\underset{j=1}{\bigoplus}}V_{t(l_{j})}$ et de la projection dans $\overset{n}{\underset{j=1}{\bigoplus}}V_{t(l_{j})}/\Img(\widetilde h)$.

\end{defi}

\begin{prop}
	L'application $F_{\beta}^{+}:\mathscr{L}(\Gamma,\Lambda)\rightarrow\mathscr{L}(\Gamma,\sigma_{\beta}\Lambda)$ est un foncteur. De même l'application $F_{\alpha}^{-}:\mathscr{L}(\Gamma,\Lambda)\rightarrow\mathscr{L}(\Gamma,\sigma_{\alpha}\Lambda)$ est un foncteur.
\end{prop}

\begin{proof}
	On se place dans le même cadre que lors des définitions, ainsi on a $(V,f)\in\mathscr{L}(\Gamma,\Lambda)$ une représentation du carquois $\Gamma$. On note toujours $F_{\beta}^{+}(V,f)=(W,g)\in\mathscr{L}(\Gamma,\sigma_{\beta}\Lambda)$ la représentation image de $(V,f)$ par l'application $F_{\beta}^{+}$. On a déjà la définition de $F_{\beta}^{+}$ sur les objets des catégories $\mathscr{L}(\Gamma,\Lambda)$ et $\mathscr{L}(\Gamma,\sigma_{\beta}\Lambda)$, et pour tout $(\gamma,l)\in\Gamma_{0}\times\Gamma_{1}$, on a bien que $W_{\gamma}$ est un espace vectoriel, et que $g_{l}$ est une application linéaire. Il nous reste donc à définir $F_{\beta}^{+}$ sur les morphismes de représentations pour obtenir son caractère fonctoriel. Ainsi, soient $(V,f)$ et $(V',f')$ deux représentations de $\Gamma$, on note encore $l_{1},\dots,l_{n}\in\Gamma^{\beta}$ les arêtes en direction du puits $\beta$, et soit $\varphi=(\varphi_{\gamma})_{\gamma\in\Gamma_{0}}$ un morphisme de représentation entre $(V,f)$ et $(V',f')$. On note $(W,g)$ et $(W',g')$ les représentations images par $F_{\beta}^{+}$ de $(V,f)$ et $(V',f')$. On note également $\psi=(\psi_{\gamma})_{\gamma\in\Gamma_{0}}$ le morphisme image de $\varphi$ par $F_{\beta}^{+}$, qu'il nous faut encore définir. On pose donc :
	\[	
		\psi_{\beta}(v_{1},\dots,v_{n})=(\varphi_{s(l_{1})}(v_{1}),\dots,\varphi_{s(l_{n})}(v_{n}))
	\]
	\[
		\forall \gamma\neq\beta,\; \psi_{\gamma}=\varphi_{\gamma}.
	\]
	Assurons nous tout d'abord du caractère bien défini de $\psi$. Pour tout $\gamma\neq\beta$, on a $V_{\gamma}=W_{\gamma}$, $V'_{\gamma}=W'_{\gamma}$ et $\psi_{\gamma}=\varphi_{\gamma}$, le morphisme $\psi$ est donc bien défini pour tout sommet $\gamma\neq\beta$. Il nous reste à vérifier que $\psi_{\beta}$ est bien défini. Soit $(v_{1},\dots,v_{n})\in\Ker(h)$, il nous faut voir que $\psi_{\beta}(v_{1},\dots,v_{n})\in\Ker(k)$, où
	\[
\begin{array}{lccc}
	h : & \overset{n}{\underset{j=1}{\bigoplus}}V_{s(l_{j})}&\rightarrow & V_{\beta} \\ 
	& (v_{1},\dots,v_{n})&\mapsto & \underset{i=1}{\overset{n}{\sum}}f_{l_{i}}(v_{i})\\
	&&&\\
	k : & \overset{n}{\underset{j=1}{\bigoplus}}V'_{s(l_{j})}&\rightarrow & V'_{\beta} \\ 
	& (v'_{1},\dots,v'_{n})&\mapsto & \underset{i=1}{\overset{n}{\sum}}f'_{l_{i}}(v'_{i}).\\
\end{array}
	\]
Ainsi :
\[
\begin{array}{rl}
	k(\psi_{\beta}(v_{1},\dots,v_{n})) & = \underset{i=1}{\overset{n}{\sum}}f'_{l_{i}}(\varphi_{s(l_{i})}(v_{i})) \\ 
	& = \underset{i=1}{\overset{n}{\sum}} \varphi_{\beta}(f_{l_{i}}(v_{i}))\\
	& = \varphi_{\beta}(\underset{i=1}{\overset{n}{\sum}}f_{l_{i}}(v_{i}))\\
	& = \varphi_{\beta}(0) \\
	& = 0 \\
\end{array}
\]
donc $\psi_{\beta}$ est bien définie. Voyons désormais que $\psi$ est bien un morphisme de représentations entre $(W,g)$ et $(W',g')$. Comme pour tout $\gamma\neq\beta$, on a $\psi_{\gamma}=\varphi_{\gamma}$, et pour tout $l\notin\Gamma^{\beta}$, on a $g_{l}=f_{l}$ et $g'_{l}=f'_{l}$, il vient facilement que $g'_{l_{i}}\psi_{s(l_{i})}=\psi_{t(l_{i})}g_{l_{i}}$ dès lors que $l\notin\Gamma^{\beta}$. Regardons maintenant le cas où $l\in\Gamma^{\beta}$, il existe alors $i\in[\![1,n]\!]$ tel que $l=l_{i}$, et on a $t(l_{i})=\beta$. Soit $(v_{1},\dots,v_{n})\in W_{\beta}$, il vient, d'une part :

\[
\begin{array}{ll}
	g'_{l_{i}}\psi_{\beta}(v_{1},\dots,v_{n})&=g'_{l_{i}}(\varphi_{s(l_{1})}(v_{1}),\dots,\varphi_{s(l_{n})}(v_{n}))\\
	&=\varphi_{s(l_{i})}(v_{i})
\end{array}
\]
et d'autre part :
\[
\begin{array}{ll}
	\psi_{s(l_{i})}g_{l_{i}}(v_{1},\dots,v_{n})&=\psi_{s(l_{i})}(v_{i})\\
	&=\varphi_{s(l_{i})}(v_{i}).
\end{array}
\]
Ainsi, on a bien que $\forall l \in\Gamma_{1},\;g'_{l}\psi_{t(l)}=\psi_{s(l)}g_{l}$, ce qui prouve que $\psi$ est un morphisme de représentations de $(W,g)$ vers $(W',g')$. Il nous reste donc à voir que $F_{\beta}^{+}$ respecte la composition des morphismes et qu'il envoie l'identité sur l'identité. À cet effet, posons $\Id = (\Id_{V_{\gamma}})_{\gamma\in\Gamma_{0}}$ le morphisme idendité de $(V,f)$ dans $(V,f)$, c'est à dire que pour tout sommet $\gamma\in\Gamma_{0}$, $\Id_{V_{\gamma}}$ est l'application identité de $V_{\gamma}$ dans lui-même. On a :
\[
	\forall \gamma\neq\beta,\;F_{\beta}^{+}(\Id)_{\gamma}=\Id_{V_{\gamma}}
\]
\[
	F_{\beta}^{+}(\Id)_{\beta}=(\Id_{V_{s(l_{1})}},\dots,\Id_{V_{s(l_{n})}})=\Id_{\Ker(h)}.
\]
Le morphisme identité de $(V,f)$ est donc bien envoyé sur le morphisme identité de $F_{\beta}^{+}(V,f)$. Posons désormais $(V,f)$,$(V',f')$ et $(V'',f'')$ trois représentations de $(\Gamma,\Lambda)$, $\varphi$ (respectivement $\varphi '$) un morphisme de représentation de $(V,f)$ dans $(V',f')$ (respectivement de $(V',f')$ dans $(V'',f'')$), ainsi que $(W,g)$, $(W',g')$, $(W'',g'')$, 
%% à mon avis c'est une erreur de les nommer gamma et gamma'
 $\psi$ et $\psi '$ leurs images par $F_{\beta}^{+}$. Il vient :
\[
	\forall \gamma\neq\beta,\;F_{\beta}^{+}(\varphi')_{\gamma}F_{\beta}^{+}(\varphi)_{\gamma}=\varphi_{\gamma}'\varphi_{\gamma}=F_{\beta}^{+}(\varphi'\varphi)_{\gamma}
\]
\[
	\begin{array}{ll}
		\forall (v_{1},\dots,v_{n})\in\Ker(h),& \\
		F_{\beta}^{+}(\varphi')_{\beta}F_{\beta}^{+}(\varphi)_{\beta}(v_{1},\dots,v_{n})&=F_{\beta}^{+}(\varphi')_{\beta}(\varphi_{s(l_{1})}(v_{1}),\dots,\varphi_{s(l_{n})}(v_{n}))\\
		&=(\varphi'_{s(l_{1})}(\varphi_{s(l_{1})}(v_{1})),\dots,\varphi'_{s(l_{n})}(\varphi_{s(l_{n})}(v_{n})))\\
		&=F_{\beta}^{+}(\varphi'\varphi)_{\beta}(v_{1},\dots,v_{n})

	\end{array}
\]
Ainsi la composition de deux morphismes est bien respectée par $F_{\beta}^{+}$, et cela conclue la preuve pour cette application.

On a déjà défini l'action de $F_{\alpha}^{-}$ sur les représentations de la catégorie $\mathscr L(\Gamma,\Lambda)$ à valeurs dans la catégorie $\mathscr L(\Gamma,\sigma_{\alpha}\Lambda)$. Il s'agit maintenant de définir l'action de $F_{\alpha}^{-}$ sur les morphismes de représentations. Soit $(V,f)$ et $(V',f')$ deux représentations et $\varphi = (\varphi_{\gamma})_{\gamma \in \Gamma_{0}}$ un morphisme entre ces deux représentations dans $\mathscr L(\Gamma,\Lambda)$. Posons pout tout $\gamma \neq \alpha, F_{\alpha}^{-}(\varphi)_{\gamma} = \varphi_{\gamma}$, car $F_{\alpha}^{-}$ laisse inchangé les espaces vectoriels associés aux sommets différents de $\alpha$. On définie alors l'action de  $F^{-}_{\alpha}(\varphi)_{\alpha}$ sur $\oplus_{l \in \Gamma^{\alpha}} V_{t(l)} / \Img \widetilde{h}$  dans $\oplus_{l \in \Gamma^{\alpha}} V'_{t(l)} / \Img \widetilde{h'}$ avec $\widetilde{h'}$ la fonction de $V_{\alpha}$ dans  $\oplus_{l \in \Gamma^{\alpha}} V'_{t(l)}$ qui à $v$ associe $(f'_{1}(v), \dots , f'_{n}(v))$ par : 
\[
\forall(y_{1}, \dots, y_{n})+ \Img \widetilde{h} \in  \oplus_{ i=1}^{n} V_{t(l_{i})} / \Img \widetilde{h}, \\ 
F_{\alpha}^{-}(\varphi)_{\alpha}((y_{1}, \dots, y_{n}) +\Img \widetilde{h}) = (\varphi_{t(l_{1})}(y_{1}), \dots , \varphi_{t(l_{n})}(y_{n})) + \Img \widetilde{h'}
\]

Vérifions alors que cette définition a un sens, soit $(y_{1}, \dots, y_{n})$ et $(z_{1}, \dots , z_{n})$ deux éléments de $\oplus_{i=1}^{n} V_{t(l_{i})}$ ayant la même classe dans $\oplus_{i=1}^{n} V_{t(l_{i})} / \Img \widetilde{h}$ c'est à dire qu'il existe $v \in V_{\alpha}$ tel que  $(y_{1}, \dots, y_{n}) = (z_{1}, \dots , z_{n}) + (f_{t(l_{1})}(v), \dots, f_{t(l_{n})}(v))$.
\[
\begin{array}{ll}
  \left(\varphi_{t(l_{1})}(y_{1}), \dots, \varphi_{t(l_{n})}(y_{n}) \right) &= \left((\varphi_{t(l_{1})}(z_{1} + f_{t(l_{1})}(v)), \dots, \varphi_{t(l_{n})}(z_{n} + f_{t(l_{n})}(v))\right) \\
  &= \left(\varphi_{t(l_{1})}(z_{1}), \dots, \varphi_{t(l_{n})}(z_{n})\right) + \left(\varphi_{t(l_{1})}(f_{t(l_{1})}(v)), \dots, \varphi_{t(l_{n})}(f_{t(l_{n})}(v)) \right)\\
&= \left(\varphi_{t(l_{1})}(z_{1}), \dots, \varphi_{t(l_{n})}(z_{n})\right) + \left(f'_{t(l_{1})}((\varphi_{\alpha}(v)), \dots, f'_{t(l_{n})}((\varphi_{\alpha}(v)) \right)
\end{array}
\]

 car pour tout $i \in [\![1,n]\!]$, on a $\varphi_{t(l_{i})}  f_{l_{i}} = f'_{l_{i}}  \varphi_{\alpha} $, donc finalement $F_{\alpha}^{-}(\varphi_{\alpha})$ est bien définie.


Soit $g_{t(l_{i})}$ comme définie ci-dessus et  pour tout  $i\in[\![1,n]\!],g'_{l_{i}}$ la composition de l'inclusion naturelle de $V'_{t(l_{i})}$ dans $\oplus_{i = 1}^{n} V'_{t(l_{i})}$ avec la projection de $\oplus_{l \in \Gamma^{\alpha}} V'_{t(l)}$ dans $\oplus_{l \in \Gamma^{\alpha}} V'_{t(l)} / \Img \widetilde{h'}$. Montrons alors que pour tout $i \in [\![1,n]\!]$ et $x \in V_{t(l_{i})}$, on ait $F^{-}_{\alpha}(\varphi_{\alpha})  g_{l_{i}}(x)  = g'_{l_{i}}  \varphi_{l_{i}}(x)$ pour vérifer que $F_{\alpha}^{-}(\varphi)$ est bien un morphisme de représentation dans $\mathscr L(\Gamma,\sigma_{\alpha}\Lambda)$. Et en effet, 
\[
\begin{array}{ll}
  F^{-}_{\alpha}(\varphi_{\alpha})  g_{l_{i}}(x) &= F^{-}_{\alpha}(\varphi_{\alpha})((0, \dots, x, \dots, 0) + \Img \widetilde{h}) \\
                                                    &= (0, \dots, \varphi_{t(l_{i})}(x),\dots,0) +\Img \widetilde{h'} \\
                                                    &= g'_{l_{i}}  \varphi_{t(l_{i})}(x)
\end{array}
\] 

Montrons dès à présent que la composition des morphismes de représentations est respectée par $F^{-}_{\alpha}$. Soit $(V,f)$, $(V',f')$ et $(V'',f'')$ trois représentations de $\mathscr L(\Gamma,\Lambda)$ et $(W,g)$, $(W',g')$ et $(W'',g'')$ leur image par $F^{-}_{\alpha}$, et $\varphi$ ( respectivement $\varphi'$) un morphisme de $(V,f)$ dans $(V',f')$ (respectivement $(V',f')$ dans $(V'',f'')$). Ainsi, pour tout $\gamma \neq \alpha$, on a par définition de $F^{-}_{\alpha}$, $F^{-}_{\alpha}(\varphi' \varphi)_{\gamma} = (\varphi' \varphi)_{\gamma} = \varphi'_{\gamma} \varphi_{\gamma} = F^{-}_{\alpha}(\varphi')_{\gamma}  F^{-}_{\alpha}(\varphi)_{\gamma}$.

\[
\begin{array}{rl}
  \forall (y_{1}, \dots, y_{n}) + \Img \widetilde{h} \in \oplus_{i=1}^{n} V_{t(l_{i})} / \Img \widetilde{h},& \\
  F^{-}_{\alpha}(\varphi' \varphi)_{\alpha}((y_{1}, \dots, y_{n} + \Img \widetilde{h}) &= ((\varphi' \varphi)_{t(l_{1})}(y_{1}), \dots, (\varphi' \varphi)_{t(l_{n})}(y_{n})) + \Img \widetilde{h''} \\
                                                                                                    &= F^{-}_{\alpha}(\varphi')_{\alpha} F^{-}_{\alpha}(\varphi)_{\alpha}((y_{1}, \dots, y_{n}) + \Img \widetilde{h})
\end{array}
\]
 On a bien montré que $F^{-}_{\alpha}(\varphi' \varphi) = F^{-}_{\alpha}(\varphi') F^{-}_{\alpha}(\varphi)$.

 Il reste à voir $\Id$ le morphisme identité d'une représentation $(V,f)$ est envoyé sur le morphisme identité de $F^{-}_{\alpha}((V,f))$ par $F^{-}_{\alpha}$. Pour tout $\gamma \neq \alpha$, on a $F^{-}_{\alpha}(\Id)_{\gamma} = Id_{\gamma}$ et : 

\[
\begin{array}{rl}
  \forall (y_{1}, \dots, y_{n}) + \Img \widetilde{h} \in \oplus_{i=1}^{n} V_{t(l_{i})} / \Img \widetilde{h},& \\
F^{-}_{\alpha}(\Id)_{\alpha}((y_{1}, \dots, y_{n}) + \Img \widetilde{h}) &= (y_{1}, \dots, y_{n}) + \Img \widetilde{h}
\end{array}
\]
Donc on a le résultat.
\end{proof}

\begin{defi}
%% j'aurais mis beta un puit perso et j'aurais utiliser alpha ^^
% En fait je voulais mettre un puits au début, je me suis planté, mais du coup c'est bien beta si on veut être raccord avec les notations des russes !
	Soit $\beta\in\Gamma_{0}$ un puits suivant l'orientation $\Lambda$, alors $\beta$ est une source suivant l'orientation $\sigma_{\beta}\Lambda$. On peut donc définir le foncteur : $F_{\beta}^{-}F_{\beta}^{+}:\mathscr{L}(\Gamma,\Lambda)\rightarrow\mathscr{L}(\Gamma,\Lambda).$ Soit $(V,f)\in\mathscr{L}(\Gamma,\Lambda)$, on note $(Z,e)=F_{\beta}^{-}F_{\beta}^{+}(V,f)$ l'image de $(V,f)$ par $F_{\beta}^{-}F_{\beta}^{+}$ et on construit alors une application $i_{V}^{\beta}:(Z,e)\rightarrow (V,f)$ de la manière suivante :
	\[
		\forall \gamma\neq\beta,\;Z_{\gamma}=V_{\gamma}\text{, on pose donc : }(i_{V}^{\beta})_{\gamma}=\Id_{V_{\gamma}}.
	\]
	Pour $(i_{V}^{\beta})_{\beta}$, on prend l'application : 
	\[
	\begin{array}{ccc}
		Z_{\beta}=\underset{l\in\Gamma^{\beta}}{\oplus} V_{s(l)}/\Img \widetilde{h}=\underset{l\in\Gamma^{\beta}}{\oplus} V_{s(l)}/\Ker h & \rightarrow & V_{\beta} \\
		(v_{1},\dots,v_{n})+\Ker h & \mapsto & \sum_{i=1}^{n}f_{l_{i}}(v_{i})
	\end{array}
	\]
	où $\Gamma^{\beta}=\left\{ l_{1},\dots,l_{n} \right\}$.

        Soit $\alpha\in\Gamma_{0}$ une source suivant l'orientation $\Lambda$, alors $\alpha$ est un puits suivant l'orientation $\sigma_{\alpha}\Lambda$. On peut donc définir le foncteur : $F_{\alpha}^{+}F_{\alpha}^{-}:\mathscr{L}(\Gamma,\Lambda)\rightarrow\mathscr{L}(\Gamma,\Lambda).$ Soit $(V,f)\in\mathscr{L}(\Gamma,\Lambda)$, on note $(Z,e)=F_{\alpha}^{+}F_{\alpha}^{-}(V,f)$ l'image de $(V,f)$ par $F_{\alpha}^{+}F_{\alpha}^{-}$ et on construit alors une application $p_{V}^{\alpha}:  (V,f)\rightarrow (Z,e)$ de la manière suivante :
	\[
		\forall \gamma\neq\alpha,\;Z_{\gamma}=V_{\gamma}\text{, on pose donc : }(p_{V}^{\alpha})_{\gamma}=\Id_{Z_{\gamma}}.
	\]
	Pour $(p_{V}^{\alpha})_{\alpha}$, on prend l'application : 
	\[
	\begin{array}{ccc}
		 V_{\alpha} & \rightarrow & Z_{\alpha} \\
		v & \mapsto & (f_{l_{1}}(v), \dots, f_{l_{n}}(v))
	\end{array}
	\]
	où $\Gamma^{\alpha}=\left\{ l_{1},\dots,l_{n} \right\}$.
\end{defi}
\begin{prop}
	L'application $i_{V}^{\beta}$ est un morphisme de $(Z,e)$ dans $(V,f)$ et l'application $p_{V}^{\alpha}$ est un morphisme de $(V,f)$ dans $(Z,e)$.
\end{prop}
\begin{proof}
	Notons $(W,g)=F_{\beta}^{+}(V,f)$ l'image de $(V,f)$ par $F_{\beta}^{+}$. Dans la définition de $(i_{V}^{\beta})_{\beta}$, on a implicitement déclaré que $\Ker h=\Img\widetilde h$, nous allons le prouver. Considérons la suite d'applications suivante :
	\[
		W_{\beta}\overset{\widetilde h}{\longrightarrow}\underset{l\in\Gamma^{\beta}}{\oplus}V_{s(l)}\overset{h}{\longrightarrow}V_{\beta},
	\]
on a alors :
\[
	\begin{array}{lll}
		\Img \widetilde h &=& \left\{ (g_{l_{1}}(v_{1}),\dots,g_{l_{n}}(v_{n})) \;|\; (v_{1},\dots,v_{n})\in W_{\beta} \right\}\\
		&=& \left\{ (v_{1},\dots,v_{n})\;|\;(v_{1},\dots,v_{n})\in\Ker h \right\}\\
		&=& \Ker h.
\end{array}
\]
Car $W_{\beta}=\Ker h$ par définition et pour tout $i\in[\![1,n]\!]$, $g_{l_{i}}$ est la composée de l'injection $\Ker h \hookrightarrow \oplus_{l\in\Gamma^{\beta}}V_{s(l)}$ avec la projection $\oplus_{l\in\Gamma^{\beta}}V_{s(l)}\twoheadrightarrow V_{s(l_{i})}$. Prouvons désormais que $i_{V}^{\beta}$ est un morphisme, c'est-à-dire que le diagramme suivant commute :
\[
	\begin{tikzpicture}
	\matrix (m) [matrix of math nodes,row sep=3em,column sep=4em,minimum width=2em]
	  {
		  Z_{s(l)} & Z_{t(l)} \\
		  V_{s(l)} & V_{t(l)} \\};
	\path[-stealth]
	(m-1-1) edge node [left] {$(i_{V}^{\beta})_{s(l)}$} (m-2-1)
		edge node [above] {$e_{l}$} (m-1-2)
	(m-2-1) edge node [below] {$f_{l}$} (m-2-2)
	(m-1-2) edge node [right] {$(i_{V}^{\beta})_{t(l)}$} (m-2-2);
	\end{tikzpicture}
\]
ou encore que $\forall l\in\Gamma_{1}$, on a $(i_{V}^{\beta})_{t(l)}e_{l}=f_{l}(i_{V}^{\beta})_{s(l)}$. Commençons par prendre $l\notin\Gamma^{\beta}$, on a alors $(i_{V}^{\beta})_{s(l)}=\Id_{V_{s(l)}}$, $(i_{V}^{\beta})_{t(l)}=\Id_{V_{t(l)}}$, et $e_{l}=f_{l}$, d'où l'égalité souhaitée. Voyons maintenant le cas où $l\in\Gamma^{\beta}$, on alors $t(l)=\beta$. Vérifions dans un premier temps que l'application $(i_{V}^{\beta})_{\beta}$ est bien définie. Pour cela prenons deux représentants de la même classe dans $Z_{\beta}$ : $(v_{1},\dots,v_{n})$ et $(v_{1}',\dots,v_{n}')$, et vérifions qu'ils ont la même image par $(i_{V}^{\beta})_{\beta}$. On a $(v_{1}-v_{1}',\dots,v_{n}-v_{n}')\in\Ker h$, d'où :
\[
\begin{array}{>{\dps}r>{\dps}l>{\dps}l}
	0&=&\sum_{i=1}^{n}f_{l_{i}}(v_{i}-v_{i}')\\
	&=& \sum_{i=1}^{n}f_{l_{i}}(v_{i})-\sum_{i=1}^{n}f_{l_{i}}(v_{i}')\\
	&=& (i_{V}^{\beta})_{\beta}((v_{1},\dots,v_{n})+\Ker h)-(i_{V}^{\beta})_{\beta}((v_{1}',\dots,v_{n}')+\Ker h).

\end{array}
\]
Et on a bien $(i_{V}^{\beta})_{\beta}((v_{1},\dots,v_{n})+\Ker h)=(i_{V}^{\beta})_{\beta}((v_{1}',\dots,v_{n}')+\Ker h)
$, d'où le caractère bien définie de $(i_{V}^{\beta})_{\beta}$. Notons $l=l_{j}$ pour un certain $j\in[\![1,n]\!]$ et vérifions que $f_{l_{j}}\Id_{V_{s(l_{j})}}=(i_{V}^{\beta})_{\beta}e_{l_{j}}$. Soit $v_{j}\in V_{s(l_{j})}$, on a d'une part :
\[
	f_{l_{j}}\Id_{V_{s(l_{j})}}(v_{j})=f_{l_{j}}(v_{j})
\]
et d'autre part :
\[
\begin{array}{>{\dps}r>{\dps}l>{\dps}l}
	(i_{V}^{\beta})_{\beta}e_{l_{j}}(v_{j})&=& (i_{V}^{\beta})_{\beta}((0,\dots,0,v_{j},0,\dots,0)+\Ker h)\\
	&=& f_{l_{j}}(v_{j}).
\end{array}
\]
Et on a encore l'égalité souhaitée, ainsi $i_{V}^{\beta}$ est bien un morphisme.

Montrons de même que $p_{V}^{\alpha}$ est un morphisme. Commençons par montrer que $(p_{V}^{\alpha})_{\alpha}$ est bien définie, c'est à dire que son image est contenue dans $F_{\alpha}^{+}F_{\alpha}^{-}(V)$. Notons $(W,g) = F_{\alpha}^{-}(V,f)$, alors si $\Gamma^{\alpha} = \{l_{1}, \dots, l_{n}\}$, on sait pour tout $j\in[\![1,n]\!], g_{l_{j}}$ est la composition de l'inclusion naturelle de $W_{t(l_j)}$ dans $\overset{n}{\underset{i=1}{\bigoplus}}W_{t(l_{i})}$ et de la projection dans $\overset{n}{\underset{i=1}{\bigoplus}}W_{t(l_{i})}/\Img(\widetilde h)$.
\[
\begin{array}{rl}
\Ker h &= \left\{ \sum_{j=1}^{n}g_{l_{j}}(w_{j}) = 0, (w_{1}, \dots, w_{n}) \in \overset{n}{\underset{i=1}{\bigoplus}}W_{t(l_{i})} \right\} \\
       &= \left\{ (v_{1}, \dots, v_{n}) + \Img \widetilde{h} = 0, (v_{1}, \dots, v_{n}) \in \overset{n}{\underset{i=1}{\bigoplus}}V_{t(l_{i})} \right\} \\
       &= \Img \widetilde{h}\\
       &= \left\{ (f_{l_{1}(v)},\dots,f_{l_{n}}(v)), v \in V_{\alpha} \right\} \\
\end{array}
\]
Donc $p_{V}^{\alpha}$ est bien définie, on peut alors montrer que c'est un morphisme.

Soit $l \in \Gamma_{1}$, on souhaite que le diagramme suivant commute.

\[
	\begin{tikzpicture}
	\matrix (m) [matrix of math nodes,row sep=3em,column sep=4em,minimum width=2em]
	  {
		  V_{s(l)} & V_{t(l)} \\
		  Z_{s(l)} & Z_{t(l)} \\};
	\path[-stealth]
	(m-1-1) edge node [left] {$(p_{V}^{\alpha})_{s(l)}$} (m-2-1)
		edge node [above] {$f_{l}$} (m-1-2)
	(m-2-1) edge node [below] {$e_{l}$} (m-2-2)
	(m-1-2) edge node [right] {$(p_{V}^{\alpha})_{t(l)}$} (m-2-2);
	\end{tikzpicture}
\]  
Si $l \notin \Gamma^{\alpha}$ alors $(p_{V}^{\alpha})_{s(l)} = \Id_{Z_{s(l)}}$, $(p_{V}^{\alpha})_{t(l)} = \Id_{Z_{t(l)}}$ et $e_{l} = f_{l}$ donc $e_{l}(p_{V}^{\alpha})_{s(l)} = (p_{V}^{\alpha})_{t(l)}f_{l}$.
Si $l \in \Gamma^{\alpha}$ alors $s(l) = \alpha $, $(p_{V}^{\alpha})_{t(l)} = \Id_{Z_{t(l)}}$ et pour tout $v \in V_{\alpha}$,

\[
\begin{array}{rl}
  e_{l}(p_{V}^{\alpha})_{\alpha}(v)&= e_{l}(f_{l_{1}}(v), \dots, f_{l_{n}}(v)) \\
                                   &= f_{l}(v)
\end{array}
\]
Car on sait que pour tout  $e_{l}$ est  la composition de l'inclusion naturelle de $\Ker h$ dans $\overset{n}{\underset{i=1}{\bigoplus}}Z_{t(l_{i})}$ et de la projection dans $Z_{t(l)}$.


\end{proof}
\begin{lm}
\label{lemmecrucial}
	On a les résultats suivants :
	\begin{enumerate}
		\item $F_{\alpha}^{\pm}((V_{1},f)\oplus (V_{2},g))=F_{\alpha}^{\pm}(V_{1},f)\oplus F_{\alpha}^{\pm}(V_{2},g)$ ;
		\item $p_{V}^{\alpha}$ est surjectif et $i_{V}^{\beta}$ est injectif ;
		\item \begin{enumerate}
				\item si $i_{V}^{\beta}$ est un isomorphisme, alors on a les formules :
					\[
\begin{array}{>{\dps}r>{\dps}l>{\dps}l}
	\forall\gamma\neq\beta,\;\dim F_{\beta}^{+}(V,f)_{\gamma}&=& \dim V_{\gamma}\\
	\dim F_{\beta}^{+}(V,f)_{\beta}&=& -\dim V_{\beta}+\sum_{l\in\Gamma^{\beta}}\dim V_{s(l)}
\end{array}
					\]
				\item si $p_{V}^{\alpha}$ est un isomorphisme, alors on a les formules :
	\[
\begin{array}{>{\dps}r>{\dps}l>{\dps}l}
	\forall\gamma\neq\alpha,\;\dim F_{\alpha}^{-}(V,f)_{\gamma}&=& \dim V_{\gamma}\\
	\dim F_{\alpha}^{-}(V,f)_{\alpha}&=& -\dim V_{\alpha}+\sum_{l\in\Gamma^{\alpha}}\dim V_{t(l)};
\end{array}
					\]
			\end{enumerate}
		\item $\forall \gamma\neq\alpha,\;(\Ker p_{V}^{\alpha})_{\gamma}=\left\{ 0 \right\}$ et $\forall \gamma\neq\beta,\;(V/\Img i_{V}^{\beta})_{\gamma}=\left\{ 0 \right\}$ ;
		\item si $V$ est de la forme $F_{\alpha}^{+}(W,g)$ (respectivement $F_{\beta}^{-}(W,g)$), alors $p_{V}^{\alpha}$ ($i_{V}^{\beta}$) est un isomorphisme ;
		\item $V$ est isomorphe à $F_{\beta}^{-}F_{\beta}^{+}(V,f)\oplus V/\Img i_{V}^{\beta}$, de même $V$ est isomorphe à $F_{\alpha}^{+}F_{\alpha}^{-}(V,f)\oplus \Ker p_{V}^{\alpha}$.
	\end{enumerate}
\end{lm}
\begin{proof}
	Prouvons le point $1$. Soit $\beta\in\Gamma_{0}$ un puits de $\Gamma$ suivant l'orientation $\Lambda$, soient de plus $(V,f)$ et $(V',f')$ deux représentations de $(\Gamma,\Lambda)$. On veut montrer que $F_{\beta}^{+}((V,f)\oplus(V',f'))=F_{\beta}^{+}(V,f)\oplus F_{\beta}^{+}(V',f')$. Il vient :
	\[
\begin{array}{>{\dps}r>{\dps}l>{\dps}l}
	\forall\gamma\neq\beta,\;F_{\beta}^{+}((V,f)\oplus(V',f'))_{\gamma}&=&((V,f)\oplus(V',f'))_{\gamma}\\
	&=& V_{\gamma}\oplus V_{\gamma}'\\
	&=&  F_{\beta}^{+}(V,f)_{\gamma}\oplus F_{\beta}^{+}(V',f')_{\gamma}\\
	&&\\
	F_{\beta}^{+}((V,f)\oplus(V',f'))_{\beta}&=&\Ker(h\oplus h')\\
	&=& \Ker h \oplus \Ker h'\\
	&=&  F_{\beta}^{+}(V,f)_{\beta}\oplus F_{\beta}^{+}(V',f')_{\beta},
\end{array}
	\]
où :
\[
\begin{array}{lccc}
	h : & \overset{n}{\underset{j=1}{\bigoplus}}V_{s(l_{j})}&\rightarrow & V_{\beta} \\ 
	& (v_{1},\dots,v_{n})&\mapsto & \underset{i=1}{\overset{n}{\sum}}f_{l_{i}}(v_{i})\\
	&&&\\
	h' : & \overset{n}{\underset{j=1}{\bigoplus}}V'_{s(l_{j})}&\rightarrow & V'_{\beta} \\ 
	& (v'_{1},\dots,v'_{n})&\mapsto & \underset{i=1}{\overset{n}{\sum}}f'_{l_{i}}(v'_{i}).\\
\end{array}
	\]
	et où $\Gamma^{\beta}=\left\{ l_{1},\dots,l_{n} \right\}$. Ainsi on a bien l'égalité annoncée.

Supposons maintenant que $\alpha$ est un source dans $(\Gamma,\Lambda)$, $(V,f)$ et $(V',f')$ deux représentations de $\mathscr L(\Gamma,\Lambda)$, on a : 

\[
\begin{array}{>{\dps}r>{\dps}l>{\dps}l}
	\forall\gamma\neq\alpha,\;F_{\alpha}^{-}((V,f)\oplus(V',f'))_{\gamma}&=&((V,f)\oplus(V',f'))_{\gamma}\\
	&=& V_{\gamma}\oplus V_{\gamma}'\\
	&=&  F_{\alpha}^{-}(V,f)_{\gamma}\oplus F_{\alpha}^{-}(V',f')_{\gamma}\\
	&&\\
	F_{\alpha}^{-}((V,f)\oplus(V',f'))_{\alpha}&=&\oplus_{i=1}^{n}\left(V_{l_{i}} \oplus V'_{l_{i}}\right)/ \Img \widetilde{h}_{\oplus} \\
	&\simeq& \left( \oplus_{i=1}^{n}V_{t(l_{i})} / \Img \widetilde{h} \right) \oplus  \left( \oplus_{i=1}^{n}V'_{t(l_{i})} / \Img \widetilde{h'} \right)\\
	&=&  F_{\alpha}^{-}(V,f)_{\alpha}\oplus F_{\alpha}^{-}(V',f')_{\alpha}\\
\end{array}
\] 
où :
\[
\begin{array}{lccc}
	\widetilde{h} : & V_{\alpha} &\rightarrow & \overset{n}{\underset{j=1}{\bigoplus}}V_{t(l_{j})} \\ 
	& v&\mapsto & \left(f_{l_{1}}(v), \dots, f_{l_{n}}\right)\\
	&&&\\
	\widetilde{h'} : & V'_{\alpha} &\rightarrow & \overset{n}{\underset{j=1}{\bigoplus}}V'_{t(l_{j})} \\ 
	& v'&\mapsto & \left(f'_{l_{1}}(v'), \dots, f'_{l_{n}}(v')\right)\\
	&&&\\
	\widetilde{h}_{\oplus} : &V_{\alpha} \oplus V'_{\alpha} &\rightarrow & \overset{n}{\underset{j=1}{\bigoplus}}(V_{t(l_{j})} \oplus V'_{t(l_{j})}) \\ 
	& (v,v') &\mapsto & \left((f_{l_{1}}(v),f'_{l_{1}}(v')), \dots, (f_{l_{n}}(v),f'_{l_{n}}(v'))\right)\\
\end{array}
	\]


	Prouvons désormais le point $2$. Dire que le morphisme $i_{V}^{\beta}$ est injectif signifie que pour tout $\gamma\in\Gamma_{0}$, l'application $(i_{V}^{\beta})_{\gamma}$ est injective. Comme $\forall\gamma\neq\beta,\;(i_{V}^{\beta})_{\gamma}=\Id_{V_{\gamma}}$, l'injectivité de $(i_{V}^{\beta})_{\gamma}$ est assurée pour tout $\gamma\neq\beta$. Regardons maintenant le cas $\gamma=\beta$. Soient $a=(v_{1},\dots,v_{n})+\Ker h$ et $b=(v_{1}',\dots,v_{n}')+\Ker h$ tels que $(i_{V}^{\beta})_{\beta}(a)=(i_{V}^{\beta})_{\beta}(b)$. On alors :
	\[
\begin{array}{>{\dps}r>{\dps}l>{\dps}l}
	\sum_{i=1}^{n}f_{l_{i}}(v_{i})&=&\sum_{i=1}^{n}f_{l_{i}}(v'_{i})\\
	\sum_{i=1}^{n}f_{l_{i}}(v_{i}-v_{i}')&=&0
\end{array}
	\]
	d'où $(v_{1}-v_{1}',\dots,v_{n}-v_{n}')\in\Ker h$, donc $a=b$, et on a l'injectivité de $(i_{V}^{\beta})_{\beta}$ puis de $i_{V}^{\beta}$.


Le morphisme $p_{V}^{\alpha}$ est quant à lui surjectif, car pour tout $\gamma \neq \alpha, (p_{V}^{\alpha})_{\gamma} = \Id_{V_{\gamma}}$ est surjective et $ \Img (p_{V}^{\alpha})_{\alpha} = \Img \widetilde{h}$ donc $ (p_{V}^{\alpha})_{\alpha}$ est surjective.  

	Prouvons maintenant le point $3$. Supposons que $i_{V}^{\beta}$ est un isomorphisme, en particulier, $(i_{V}^{\beta})_{\beta}$ est un isomorphisme et on a $\oplus_{l\in\Gamma^{\beta}}V_{s(l)}/\Ker h\approx V_{\beta}$. On a égalité entre les dimensions d'espaces vectoriels isomorphes, et ainsi il vient :
	\[
\begin{array}{>{\dps}r>{\dps}l>{\dps}l}
	\dim(\oplus_{l\in\Gamma^{\beta}}V_{s(l)}/\Ker h) &=& \dim V_{\beta}\\
	\sum_{l\in\Gamma^{\beta}}\dim V_{s(l)}-\dim\Ker h &=& \dim V_{\beta}\\
	\sum_{l\in\Gamma^{\beta}}\dim V_{s(l)}-\dim V_{\beta} &=& \dim\Ker h\\
	\sum_{l\in\Gamma^{\beta}}\dim V_{s(l)}-\dim V_{\beta} &=& \dim F_{\beta}^{+}(V,f)_{\beta}.\\
\end{array}
	\]
	Comme $\forall \gamma\neq\beta,\;F_{\beta}^{+}(V,f)_{\gamma}=V_{\gamma}$, on a immédiatement $\forall \gamma\neq\beta,\;\dim F_{\beta}^{+}(V,f)_{\gamma}=\dim V_{\gamma}$.

        Si $p_{V}^{\alpha}$ est un isomorphisme alors : 
        \[
        \forall \gamma \neq \beta, \dim F_{\alpha}^{-}(V)_{\gamma} = \dim V_{\gamma}
        \]
        Comme on a $V_{\alpha} \simeq \Img \widetilde{h}$,
        \[
        \begin{array}{rl}
        \dim F_{\alpha}^{-}(V)_{\alpha} &= \dim\left( \oplus_{i=1}^{n} V_{l_{i}} / \Img \widetilde{h} \right) \\ \\
                                        &= \dim\left( \oplus_{i=1}^{n} V_{l_{i}} / V_{\alpha} \right) \\
                                        &= \sum_{i=1}^{n} \dim V_{l_{i}} - \dim V_{\alpha}
\end{array}
        \]


	Prouvons à présent le point $4$. Pour tout $\gamma\neq\beta$, on a $(V/\Img i_{V}^{\beta})_{\gamma}=V_{\gamma}/\Img((i_{V}^{\beta})_{\gamma})$ et $(i_{V}^{\beta})_{\gamma}=\Id_{V_{\gamma}}$, d'où $\Img((i_{V}^{\beta})_{\gamma})=V_{\gamma}$ et ainsi $V_{\gamma}/V_{\gamma}=\left\{ 0 \right\}$.

        De même, pour tout $\gamma\neq\beta$, on a $(\Ker p_{V}^{\alpha})_{\gamma}=\Ker((p_{V}^{\alpha})_{\gamma})$ et $(p_{V}^{\alpha})_{\gamma}=\Id_{V_{\gamma}}$, d'où $\Ker((p_{V}^{\alpha})_{\gamma})=\{0\}$.

	Passons à la preuve du point 5. Pour cela, supposons que $(V,f)\in\mathscr{L}(\Gamma,\Lambda)$ est de la forme $F_{\beta}^{-}(W,g)$, il nous faut alors voir que $i_{V}^{\beta}$ est surjectif. Pour tout $\gamma\neq\beta$, on a $(i_{V}^{\beta})_{\gamma}=\Id_{V_{\gamma}}$, l'injectivité de $(i_{V}^{\beta})_{\gamma}$ est donc immédiate. Intéressons-nous maintenant au cas où $\gamma=\beta$. On se place dans l'orientation $\Lambda$ où $\beta$ est un puits et on note $\Gamma^{\beta}=\left\{ l_{1},\dots,l_{n} \right\}$, on a alors $V_{\beta}=\oplus_{i=1}^{n}V_{s(l_{i})}/\Img \widetilde{h}$. On a aussi :
	\[
\begin{array}{rccc}
	h: & \oplus_{i=1}^{n}V_{s(l_{i})} & \rightarrow & V_{\beta} \\
	& (v_{1},\dots,v_{n}) & \mapsto & \sum_{i=1}^{n}f_{l_{i}}(v_{i})\\
	&&&\\
	\text{où, pour tout }i\in[\![1,n]\!] : &&&\\
	f_{l_{i}}:& V_{s(l_{i})} & \rightarrow & V_{\beta} \\
	& v_{i} & \mapsto & (0,\dots,0,v_{i},0,\dots,0)+\Img \widetilde{h}\\
	&&&\\
	\text{d'où :} &&&\\
	h: & \oplus_{i=1}^{n}V_{s(l_{i})} & \rightarrow & V_{\beta} \\
	& (v_{1},\dots,v_{n}) & \mapsto & (v_{1},\dots,v_{n}) + \Img \widetilde{h}\\
\end{array}
	\]
	On a alors immédiatement que $\Ker h = \Img \widetilde{h}$, et on a
	\[
		\begin{array}{rccc}
			(i_{V}^{\beta})_{\beta} : & \oplus_{i=1}^{n}V_{s(l_{i})}/\Ker h & \rightarrow & V_{\beta}\\
			& (v_{1},\dots,v_{n})+\Ker h & \mapsto & (v_{1},\dots,v_{n})+\Img \widetilde{h},\\
		\end{array}
	\]
	la surjectivité de $(i_{V}^{\beta})_{\beta}$ est donc prouvée. Ainsi $i_{V}^{\beta}$ est surjectif, puis bijectif, donc est un isomorphisme.

        De manière similaire, montrons que $p_{V}^{\alpha}$ est un isomorphisme c'est à dire que $(p_{V}^{\alpha})_{\alpha}$ est injective, dans le cas où $(V,f) = F_{\alpha}^{+}(W,g)$. Écrivons $\Gamma^{\alpha} = \{l_{1}, \dots, l_{n}\}$, on sait que $V_{\alpha} = \left\{ (v_{1}, \dots, v_{n}) \in \oplus_{i=1}^{n}  V_{t(l_{i})}\right\}$, et que $\forall j \in [\![1,n]\!] , f_{l(j)}$ est la composition de l'inclusion de $V_{\alpha}$ dans $\oplus_{i=1}^{n}V_{t(l_{i})}$ et la projection sur $V_{t(l_{j})}$. Donc soit $v= (v_{1}, \dots, v_{n}) \in V_{\alpha}$, on a $ f_{l_{j}}(v) = v_{j}$. Or si $(p_{V}^{\alpha})_{\alpha}(v) = 0 $ c'est à dire $(f_{l_{1}}(v), \dots, f_{l_{n}}(v)) = 0 $ alors $v = 0$ donc $(p_{V}^{\alpha})_{\alpha}$ est injective.


	Enfin, prouvons le point 6. Notons $\widetilde V=V/\Img i_{V}^{\beta}$, il nous faut alors prouver que $V$ est isomorphe à $F_{\beta}^{-}F_{\beta}^{+}(V)\oplus\widetilde V$. Notons également $\varphi_{\beta}':V_\beta\twoheadrightarrow \widetilde V_\beta$ la projection de $V_\beta$ dans $\widetilde V_\beta$. Cette représentation admet une section, c'est-à-dire une application $\varphi_\beta:\widetilde V_\beta\rightarrow V_\beta$ telle que $\varphi_\beta'\varphi_\beta=\Id_{\widetilde V_\beta}$ : en effet il suffit d'envoyer chaque classe d'équivalence dans $\widetilde V_\beta$ sur un unique représentant de cette classe dans $V_\beta$. On posera également, pour tout $\gamma\neq\beta,\;\varphi_\gamma=0$, ce qui fait de $\varphi=(\varphi_\gamma)_{\gamma\in\Gamma_{0}}$ un morphisme de $V/\Img i_{V}^{\beta}$ dans $V$. En effet, le diagramme suivant commute :
	\[
	\begin{tikzpicture}
	\matrix (m) [matrix of math nodes,row sep=3em,column sep=4em,minimum width=2em]
	  {
		  \widetilde V_{s(l)} & \widetilde V_{t(l)} \\
		  V_{s(l)} & V_{t(l)} \\};
	\path[-stealth]
	(m-1-1) edge node [left] {$\varphi_{s(l)}$} (m-2-1)
		edge node [above] {$\widetilde f_{l}$} (m-1-2)
	(m-2-1) edge node [below] {$f_{l}$} (m-2-2)
	(m-1-2) edge node [right] {$\varphi_{t(l)}$} (m-2-2);
	\end{tikzpicture}
\]
car on a d'une part $\forall l\in\Gamma_0,\;s(l)\neq\beta$ ($\beta$ étant un puits) et ainsi $\varphi_{s(l)}=0$ d'où $f_l\varphi_{s(l)}=0$. Et d'autre part, comme on a :
\[
\begin{array}{rccc}
	\widetilde f_{l}:&\widetilde V_{s(l)}&\rightarrow&\widetilde V_{t(l)}\\
	&v+\Img\left[(i_{V}^{\beta})_{s(l)}\right]&\mapsto&f_l(v)+\Img\left[(i_{V}^{\beta})_{t(l)}\right]
\end{array}
\]
où $\widetilde V_{s(l)}=\left\{ 0 \right\}$ car $\forall l\in\Gamma_0,\;s(l)\neq\beta$, il vient $\widetilde f_l=0$ puis $\varphi_{t(l)}\widetilde f_l=0$. Ainsi $\varphi$ est bien un morphisme. On pose désormais $\xi=(\xi_\gamma)_{\gamma\in\Gamma_0}$ où $\forall\gamma\neq\beta,\;\xi_\gamma=\Id_{V_\gamma}$ et $\xi_\beta$ est définie par :
\[
\begin{array}{rccc}
\xi_\beta:&\widetilde V_\beta\oplus F_\beta^-F_\beta^+(V,f)_\beta&\rightarrow&V_\beta\\	
&(x,y)&\mapsto&\varphi_\beta(x)+(i_V^\beta)_\beta(y).\\
\end{array}
\]
Nous allons prouver que $\xi$ est un isomorphisme. Tout d'abord, $\xi$ est bien est définie car $\forall\gamma\neq\beta,\;\xi_\gamma=\Id_{V_\gamma}$, et comme $\varphi_\beta$ et $(i_V^\beta)_\beta$ sont bien définies, on a que $\xi_\beta$ est bien définie également. Prouvons maintenant que $\xi$ est un morphisme, c'est-à-dire que le diagramme suivant commute :
\[
	\begin{tikzpicture}
	\matrix (m) [matrix of math nodes,row sep=3em,column sep=4em,minimum width=2em]
	  {
		  \left[ \widetilde V\oplus F_\beta^-F_\beta^+(V,f) \right]_{s(l)} & \left[ \widetilde V\oplus F_\beta^-F_\beta^+(V,f) \right]_{t(l)} \\
		  V_{s(l)} & V_{t(l)} \\};
	\path[-stealth]
	(m-1-1) edge node [left] {$\xi_{s(l)}$} (m-2-1)
		edge node [above] {$\widetilde f_{l}\oplus e_l$} (m-1-2)
	(m-2-1) edge node [below] {$f_{l}$} (m-2-2)
	(m-1-2) edge node [right] {$\xi_{t(l)}$} (m-2-2);
	\end{tikzpicture}
\]
où les $e_l$ sont les morphismes de $F_\beta^-F_\beta^+(V,f)$. Soit $(x,y)\in\left[ \widetilde V\oplus F_\beta^-F_\beta^+(V,f) \right]_{s(l)}$ :
\[
\begin{array}{>{\dps}r>{\dps}l>{\dps}l}
	\xi_{t(l)}(\widetilde f_l\oplus e_l(x,y))&=&\xi_{t(l)}(\widetilde f_{l}(x),e_l(y))\\ 	
	&=& \varphi_{t(l)}\widetilde f_{l}(x)+(i_V^\beta)_{t(l)} e_l(y)\\
	&=& f_l\varphi_{s(l)}(x)+f_l(i_V^\beta)_{s(l)}(y)\text{ car $\varphi$ et $i_V^\beta$ sont des morphismes}\\
	&=& f_l(\varphi_{s(l)}(x)+(i_v^\beta)_{s(l)}(y))\\
	&=& f_l\xi_{s(l)}(x,y)\\
\end{array}
\]
ainsi $\xi$ est bien un morphisme. Prouvons son injectivité. Celle-ci est immédiatement vérifiée dans le cas où $\gamma\neq\beta$, vérifions qu'elle l'est aussi pour $\gamma=\beta$. Soient $(x,y),(x',y')\in\widetilde V_\beta\oplus F_\beta^-F_\beta^+(V,f)_\beta$ tels que $\xi_\beta(x,y)=\xi_\beta(x',y')$, a fortiori on $\varphi_\beta(\xi_\beta(x,y))=\varphi_\beta(\xi_\beta(x',y'))$, d'où $x=y$. Ainsi $(i_V^\beta)_\beta(y)=(i_V^\beta)_\beta(y')$, mais par injectivité de $i_V^\beta$, il vient $y=y'$. Ainsi $(x,y)=(x',y')$ et $\xi$ est injective. Prouvons donc la sujectivité de $\xi$, une fois encore, elle est immédiate dès que $\gamma\neq\beta$, nous nous intéressons donc au cas $\gamma=\beta$. Soit donc $v\in V_\beta$, on a alors $v=\varphi_\beta\varphi_\beta'(v)+v-\varphi_\beta\varphi_\beta'(v)$, d'où il vient immédiatement que $\varphi_\beta\varphi_\beta'(v)\in\Img \varphi_\beta$. Il nous reste donc à voir que . De plus, comme $\varphi_\beta'\varphi_\beta=\Id_{\widetilde V_\beta}$, on sait que $\varphi_\beta'(v)=\varphi_\beta'\varphi_\beta\varphi_\beta'(v)$ d'où $v-\varphi_\beta\varphi_\beta'(v)\in\Img\left[ (i_V^\beta)_\beta \right]$. On a donc la surjectivité de $xi_\beta$, donc la surjectivité de $\xi$ et la bijectivité de $\xi$, qui est un morphisme. On a donc prouvé que $\xi$ est un isomorphisme, et ainsi $(V,f)$ est isomorphe à $F_\beta^-F_\beta^+(V,f)\oplus V/\Img i_V^\beta$.


Considérons la représentation $(\widetilde V, \widetilde f)$ où $\widetilde V = \Ker p_{V}^{\alpha}$ et $\widetilde{f}$ est formée :
\begin{itemize}
\item pour tout $l \notin \Gamma^{\alpha}$, $f_{l}$ est un morphisme de l'espace nul dans l'espace nul;
\item pout tout $l \in \Gamma^{\alpha}$, $f_{l}$ est un morphisme de $\widetilde V _{\alpha}$ dans l'espace nul. 
\end{itemize}

 On construit $\varphi$, le morphisme de $(V,f)$ dans $(\widetilde{V},\widetilde{f})$ défini tel que $\forall \gamma \in \Gamma_{0} \setminus \{\alpha\} $, on ait $\varphi_{\gamma}(V_{\gamma}) = \widetilde{V}_{\gamma} = \{0\}$ et $\varphi_{\alpha}(V_{\alpha}) = \widetilde{V}_{\alpha}$ c'est à dire que $\varphi_{\alpha}$ est la projection de $\widetilde{V}_{\alpha} $ dans $V_{\alpha}$. On vérifie alors que le diagramme suivant commute quel que soit $l \in \Gamma_{1}$.
\[
	\begin{tikzpicture}
	\matrix (m) [matrix of math nodes,row sep=3em,column sep=4em,minimum width=2em]
	  {
		   V_{s(l)} &  V_{t(l)} \\
		  \widetilde V_{s(l)} & \widetilde V_{t(l)} \\};
	\path[-stealth]
	(m-1-1) edge node [left] {$\varphi_{s(l)}$} (m-2-1)
		edge node [above] {$ f_{l}$} (m-1-2)
	(m-2-1) edge node [below] {$\widetilde f_{l}$} (m-2-2)
	(m-1-2) edge node [right] {$\varphi_{t(l)}$} (m-2-2);
	\end{tikzpicture}
\]


On construit alors le morphisme $\xi$ de $(V,f)$ dans $(Z,e) \oplus (\widetilde{V},\widetilde{f})$ en sommant l'action des morphismes $p^{V}_{\alpha}$ et $\varphi$, c'est à dire que pour tout $\gamma \in \Gamma_{0}, \xi_{\gamma}(V,f) = (p_{V}^{\alpha}(V,f) \oplus \varphi_{\gamma}(V,f)$. Comme $p_{V}^{\alpha}$ et $\varphi$ sont respectivement  des morphismes de $(V,f)$ dans $(Z,e)$ et de $(V,f)$ dans $(\widetilde V, \widetilde f)$, alors $\xi$ est un morphisme de $(V,f)$ dans $(Z,e) \oplus (\widetilde V, \widetilde f)$.

Il s'agit maintenant de montrer que $\xi$ est un isomorphisme c'est à dire que pour tout $\gamma \in \Gamma_{0}$, $\xi_{\gamma}$ est un isomorphisme d'espace vectoriel. Lorsque $\gamma \in \Gamma_{0} \setminus {\alpha}, (p_{V}^{\alpha})_{\gamma} = \Id_{V_{\gamma}}$ et $\varphi_{\gamma}$ est l'application nulle donc $\xi_{\gamma}$ est un isomorphime de $V_{\gamma}$ dans $V_{\gamma}\oplus \{0\}$. $\xi_{\alpha}$ est un isomorphisme, en effet il est surjectif car $p_{V}^{\alpha}$ et $\varphi_{\alpha}$ sont surjectifs d'après ce qui précède et parce que $\varphi_{\alpha}$ est une projection. $\xi_{\alpha}$ est injectif, soit $v \in V_{\alpha}$ tel que $((p_{V}^{\alpha})_{\alpha}(v),\varphi_{\alpha}(v)) = 0$ alors $v$ appartient  à $(\Ker p_{V}^{\alpha})_{\alpha}$ donc $\varphi_{\alpha}(v) =v$ d'où $v= 0$. Donc $\xi$ est bien un isomorphisme et $(V,f) \simeq (Z,e) \oplus (\widetilde V, \widetilde f)$.


\end{proof}
\begin{thm}
	\label{thm1}
	\begin{enumerate}
		Soit $(\Gamma,\Lambda)$ un carquois.
	\item Soit $\beta\in\Gamma_0 $ un puits suivant l'orientation $\Lambda$. Soit $(V,f)\in\mathscr L(\Gamma,\Lambda)$ une représentation indécomposable. On a alors deux cas possibles :
			\begin{enumerate}
				\item $V\approx L_\beta$ (où $L_\beta$ a été défini à l'exemple \ref{irreductible}) ;
				\item $F_\beta^+(V,f)$ est indécomposable, $F_\beta^-F_\beta^+(V,f)=(V,f)$ et les dimensions des espaces $F_\beta^+(V,f)_\gamma$ sont données par :
					\[
\begin{array}{>{\dps}r>{\dps}l>{\dps}l}
	\forall\gamma\neq\beta,\;\dim F_\beta^+(V,f)_\gamma&=& \dim V_\gamma\\
	\dim F_\beta^+(V,f)_\beta&=& -\dim V_\beta+\sum_{l\in\Gamma^{\beta}}\dim V_{s(l)}.
\end{array}
					\]
			\end{enumerate}
		\item Soit $\alpha\in\Gamma_0 $ un source suivant l'orientation $\Lambda$. Soit $(V,f)\in\mathscr L(\Gamma,\Lambda)$ une représentation indécomposable. On a alors deux cas possibles :
			\begin{enumerate}
				\item $V\approx L_\alpha$ (où $L_\alpha$ a été défini à l'exemple \ref{irreductible}) ;
				\item $F_\alpha^-(V,f)$ est indécomposable, $F_\alpha^+F_\alpha^-(V,f)=(V,f)$ et les dimensions des espaces $F_\alpha^-(V,f)_\gamma$ sont données par :
					\[
\begin{array}{>{\dps}r>{\dps}l>{\dps}l}
	\forall\gamma\neq\alpha,\;\dim F_\alpha^-(V,f)_\gamma&=& \dim V_\gamma\\
	\dim F_\alpha^-(V,f)_\alpha&=& -\dim V_\alpha+\sum_{l\in\Gamma^{\alpha}}\dim V_{t(l)}.
\end{array}
					\]
			\end{enumerate}
	
	\end{enumerate}
\end{thm}
\begin{proof}
	Soit $(\Gamma,\Lambda)$ un carquois et $(V,f)\in\mathscr L(\Gamma,\Lambda)$ une représentation indécomposable de ce carquois. Supposons que $\beta$ est un puits suivant l'orientation $\Lambda$. D'après le point $6$ du lemme \ref{lemmecrucial}, on a $V\approx F_\beta^-F_\beta^+(V,f)\oplus V/\Img i_v^\beta$, or $V$ est indécomposable, donc $V$ est isomorphe à l'un des deux termes.
	\begin{itemize}
		\item Cas $1$. $V\approx V/\Img i_V^\beta$ : alors $\forall \gamma\neq\beta,\;V_\gamma=\left\{ 0 \right\}$, $\forall l\in\Gamma_1,\;f_l=0$ et $V_\beta=\mathbb K$, car $(V,f)$ est indécomposable, et si on avait $\dim V_\beta>1$, on pourrait décomposer $(V,f)$ en $\dim V_\beta$ copies de $L_\beta$. Ainsi $V\approx L_\beta$.
		\item Cas $2$. $V\approx F_\beta^-F_\beta^+(V,f)$ : autrement dit, $i_V^\beta$ est un isomorphisme. Ainsi, d'après le point $3$ du lemme \ref{lemmecrucial}, on a les égalités sur les dimensions des espaces vectoriels. Montrons à présent que $(W,g)=F_\beta^+(V,f)$ est indécomposable, pour cela supposons qu'il existe $(W_1,g_1)$ et $(W_2,g_2)$ des représentations telles que $(W,g)=(W_1,g_1)\oplus(W_2,g_2)$. On a alors, d'après le point $1$ du lemme \ref{lemmecrucial} que $(V,f)\approx F_\beta^-(W_1,g_1)\oplus F_\beta^-(W_2,g_2)$. Comme $(V,f)$ est indécomposable, un des termes est la représentation nulle, supposons, sans perte de généralité, que $F_\beta^-(W_2,g_2)=0$. D'après la forme de $W$ et le point $5$ du lemme \ref{lemmecrucial}, on sait que $p_V^\beta:(W,g)\rightarrow F_\beta^+F_\beta^-(W,g)$ est un isomorphisme, or $p_V^\beta(W_2,g_2)\subset F_\beta^+F_\beta^-(W_2,g_2)=0$, d'où $(W_2,g_2)=0$. Ainsi $F_\beta^+(V)$ est indécomposable. 
	\end{itemize}
Cela termine la démonstration dans le cas où $\beta$ est un puits.
\end{proof}
\begin{defi}
	On dit qu'une suite de sommets $\beta_1,\beta_2,\dots\beta_k$ est ne suite (+)-accessible, ou une \emph{suite de puits}, suivant l'orientation $\Lambda$, si $\beta_1$ est un puits suivant l'orientation $\Lambda$, $\beta_2$ est un puits suivant l'orientation $\sigma_{\beta_1}\Lambda$, $\beta_3$ est un puits suivant l'orientation $\sigma_{\beta_2}\sigma_{\beta_1}\Lambda$, et ainsi de suite.

	On dit qu'une suite de sommets $\alpha_1,\alpha_2,\dots\alpha_k$ est une suite (-)-accessible, ou une suite de sources, suivant l'orientation $\Lambda$, si $\alpha_1$ est une source suivant l'orientation $\Lambda$, $\alpha_2$ est une source suivant l'orientation $\sigma_{\alpha_1}\Lambda$, $\alpha_3$ est une source suivant l'orientation $\sigma_{\alpha_2}\sigma_{\alpha_1}\Lambda$, et ainsi de suite.
\end{defi}
\begin{cor}
\label{cor-crucial}
Soit $(\Gamma,\Lambda)$ un carquois et $\beta_1,\dots,\beta_k$ une suite de puits.
\begin{enumerate}
	\item $\forall i\in [\![1,k]\!],\;F_{\beta_1}^-\dots F_{\beta_{i-1}}^-(L_{\beta_{i}})$ est soit la représentation nulle, soit une représentation indécomposable de $\mathscr L(\Gamma,\Lambda)$ (ici $L_{\beta_i}\in\mathscr L(\Gamma,\sigma_{\beta_{i-1}}\sigma_{\beta_{i-2}}\dots\sigma_{\beta_{1}}\Lambda)$).
	\item Soit $(V,f)\in\mathscr L(\Gamma,\Lambda)$ une représentation indécomposable, et telle que $F_{\beta_k}^+F_{\beta_{k-1}}^{+}\dots F_{\beta_1}^+(V,f)=0$, alors il existe $i\in[\![1,k]\!]$ tel que $V\approx F_{\beta_{1}}^-F_{\beta_2}^-\dots F_{\beta_{i-1}}^-(L_{\beta_i})$.
\end{enumerate}
\end{cor}
\begin{proof}
	Soit, comme dans l'énoncé, $(\Gamma,\Lambda)$ un carquois et $\beta_1,\dots,\beta_k$ une suite de puits. Soit $i\in[\![1,n]\!]$, alors $L_{\beta_i}$ est une représentation indécomposable. D'après le théorème \ref{thm1}, on sait donc que $F_{\beta_{i-1}}^-(L_{\beta_{i}})$ est une représentation indécomposable. Par récurrence, il vient que $F_{\beta_1}^-\dots F_{\beta_{i-1}}^-(L_{\beta_{i}})$ est soit nulle, soit une représentation indécomposable de $\mathscr L(\Gamma,\Lambda)$. Prenons maintenant $(V,f)\in\mathscr L(\Gamma,\Lambda)$ une représentation indécomposable, et telle que $F_{\beta_i}^+F_{\beta_{i-1}}^{+}\dots F_{\beta_1}^+(V,f)=0$. Soit $i\in[\![1,k]\!]$ le plus petit entier tel que $F_{\beta_i}^+\dots F_{\beta_1}^+(V,f)=0$, alors d'après le théorème \ref{thm1} on a $F_{\beta_{i-1}}^+\dots F_{\beta_1}^+(V,f)\approx L_{\beta_i}$, et ainsi :
	\[
		\begin{array}{>{\dps}r>{\dps}l>{\dps}l}
		F_{\beta_{i-1}}^-F_{\beta_{i-1}}^+F_{\beta_{i-2}}^+\dots F_{\beta_1}^+(V,f)&\approx& F_{\beta_{i-2}}^{+}\dots F_{\beta_1}^+(V,f)\\
		 F_{\beta_{i-1}}^-F_{\beta_{i-1}}^+F_{\beta_{i-2}}^+\dots F_{\beta_1}^+(V,f)&\approx& F_{\beta_{i-1}}^-(L_{\beta_i})
		\end{array}
	\]
	D'où $F_{\beta_{i-2}}^{+}\dots F_{\beta_{1}}^{+}(V,f)\approx F_{\beta_{i-1}}^-(L_{\beta_{i}})$, et $F_{\beta_{i-1}}^{-}(L_{\beta_i})$ n'est pas isomorphe à $L_{\beta_{i-2}}$. Sinon, en utilisant le point $5$ du lemme \ref{lemmecrucial} on aurait que $F_{\beta_{i-2}}^{+}\dots F_{\beta_{1}}^{+}(V,f)$ est la représentation nulle, ce qui entre en contradiction avec le choix de $i$. En itérant ce raisonnement il vient $V\approx F_{\beta_{1}}^-F_{\beta_2}^-\dots F_{\beta_{i-1}}^-(L_{\beta_i})$.
\end{proof}
\begin{defi}
	On dit qu'une numérotation $(\gamma_{1}, \dots, \gamma_{n})$ des sommets d'un graphe orienté $(\Gamma,\Lambda)$ est un \emph{tri topologique} des sommets de $\Gamma$ pour l'orientation $\Lambda$ si pour tout arête $l$ de $\Gamma$ telle que $s(l) = \gamma_{i}$ et $t(l) = \gamma_{j}$ avec  $i,j \in [\![1,n]\!]$ alors $i>j$.
\end{defi}
\begin{prop}
\label{tri-topo}
Soit $(\Gamma,\Lambda)$ un graphe orienté ne contenant pas de cycle orienté.
  \begin{enumerate}
  \item $(\Gamma,\Lambda)$ admet un tri topologique de ces sommets.
  \item Un tri topologique des sommets de $(\Gamma,\Lambda)$ est une suite de puits suivant l'orientation $\Lambda$.
  \end{enumerate}
\end{prop}
\begin{proof}
  Montrons d'abord qu'il existe toujours un puit dans un graphe orienté $(\Gamma,\Lambda)$ sans cycle orienté. En effet dans le cas contraire, on peut construire par reccurence une suite infinie $(\gamma_{n})_{n\in \mathbb N}$ de sommet de $\Gamma$ telle que pour tout $n \in \mathbb N$, il existe une arête orienté $l$ de $(\Gamma,\Lambda)$ avec $s(l) = \gamma_{n}$ et $t(l) = \gamma_{n+1}$. Seulement le nombre de sommets de $\Gamma$ est finie donc il existe deux entiers $k$ et $l$ distincts telle que $\gamma_{k} = \gamma_{l}$. Donc il existe un cycle orienté dans $(\Gamma, \Lambda)$.

Montrons maintenant le point 1 par reccurence sur le nombre de sommets de $\Gamma$. Si $\Gamma$ n'admet qu'un unique sommet et aucun cycle pour l'orientation $\Lambda$, c'est à dire aucune boucle alors il existe un tri topologique évident de $\Gamma$ pour l'orientation $\Lambda$.Soit $n \in \mathbb N^{*}$, supposons maintenant que tout graphe orienté $(\Gamma',\Lambda')$ sans cycle orienté ayant $n$ sommets admet un tri topologique. Soit $(\Gamma,\Lambda)$ un graphe orienté avec $n+1$ sommets sans cycle orienté. On sait qu'il existe $s$ un puit dans $\Gamma$. Considérons $(\Gamma',\Lambda')$ le graphe orienté dont les sommets sont $\Gamma_{0} \setminus \{s\}$ et les arêtes sont $\Gamma_{1} \setminus \Gamma^{s}$ avec l'orientation $\Lambda$ réduite. Par hypothèse de reccurence, il existe un tri topologique $\gamma'_{1}, \dots, \gamma'_{n}$ sur les sommets de $\Gamma'$ pour l'orientation $\Lambda'$. En posant $\gamma_{1} = s$ et pour tout $i \in [\![2,n+1]\!], \gamma_{i} = \gamma'_{i-1}$ , on obtient que $\gamma_{1}, \dots, \gamma_{n+1}$ est un tri topologique des sommets de $\Gamma$ pour l'orientation $\Lambda$. 

Montrons finalement le point 2, soit $\gamma_{1}, \dots, \gamma_{n}$ un tri topologique de $\Gamma$ pour l'orientation $\Lambda$. On remarque $\gamma_{1}$ est un puit dans $(\Gamma,\Lambda)$ car toute arête de $\Gamma^{\gamma_{1}}$ a pour destination $\gamma_{1}$. Sinon une arête de source $\gamma_{1}$ aura une destination d'indice strictement inférieure à 1, c'est impossible. Soit $i \in [\![2,n]\!]$ montrons que $\gamma_{i}$ est un puit dans $(\Gamma,\sigma_{\gamma_{i-1}} \dots \sigma_{\gamma_{1}} \Lambda)$. Soit $l \in \Gamma^{\gamma_{i}}$ une arête reliant $\gamma_{i-1}$ et $\gamma_{j}$ un autre sommmet. Si $i<j$ alors l'orientation de $l$ dans $\sigma_{\gamma_{i-1}} \dots \sigma_{\gamma_{1}} \Lambda$ est la même que dans $\Lambda$, donc la destination de $l$ est $\gamma_{i}$. Sinon l'orientation de $l$ dans $\sigma_{\gamma_{i-1}} \dots \sigma_{\gamma_{1}} \Lambda$ est l'inverse de celle dans $\Lambda$, donc la destination de $l$ est encore $\gamma_{i}$ pour l'orientation $\sigma_{\gamma_{i-1}} \dots \sigma_{\gamma_{1}} \Lambda$. Finalement $\gamma_{i}$ est toujours un puit dans $(\Gamma,\sigma_{\gamma_{i-1}} \dots \sigma_{\gamma_{1}} \Lambda)$.
\end{proof}
\begin{thm}
Soit $\Gamma$ un graphe ne contenant pas de cycle, et soient $\Lambda,\Lambda'$ deux orientations de ce graphe. 
\begin{enumerate}
	\item Il existe $\beta_1,\beta_2,\dots,\beta_k$ une suite de puits suivant, $\Lambda$ telle que $\sigma_{\beta_k}\sigma_{\beta_{k-1}}\dots\sigma_{\beta_1}\Lambda=\Lambda'$ ;
	\item Soit $\mathscr M$, (respectivement $\mathscr M'$) l'ensemble des représentations indécomposables (à isomorphisme près) de $\mathscr L(\Gamma,\Lambda)$ (respectivement de $\mathscr L(\Gamma,\Lambda')$). $\widetilde{ \mathscr M} \subset \mathscr M$ l'ensemble des classes des objets $F^{-}_{\beta_{1}}F^{-}_{\beta_{2}} \dots F^{-}_{\beta_{i-1}}(L_{\beta_{i}})$ pour $i\in [\![1,k]\!]$ et $\widetilde{ \mathscr M'} \subset \mathscr M' $ l'ensemble des classes d'objets $F^{+}_{\beta_{k}}F^{+}_{\beta_{k-1}} \dots F^{+}_{\beta_{i}}(L_{\beta_{i}})$ pour $i \in [\![1,k]\!]$, alors le foncteur $F^{+}_{\beta_{k}}F^{+}_{\beta_{k-1}} \dots F^{+}_{\beta_{1}}$ est une bijection entre $\mathscr{ M} \setminus \widetilde{\mathscr M}$ et $\mathscr{ M}' \setminus \widetilde{\mathscr{M}'}$.
\end{enumerate}
\end{thm}

\begin{proof}
  Montrons qu'il suffit de faire la preuve dans le cas où $\Lambda$ et $\Lambda'$ ne diffère que sur une arête. Pour ce faire, démontrons que l'on peut passer d'une arête à deux arêtes de différence. Le cas général se déduit alors par réccurence. 

Soit $\Lambda$ et $\Lambda'$ différentes sur $l$ et $l'$ deux arêtes distinctes de $\Gamma$. Posons $\widetilde \Lambda$ différent de $\Lambda$ sur $l$ uniquement. Alors on sait qu'il existe par hypothèse une suite de puits $\beta_{1}, \dots, \beta_{k}$ dans $( \Gamma, \Lambda)$ telle que $\sigma_{\beta_{1}}\dots\sigma_{\beta_{k}}\Lambda = \widetilde \Lambda$. De même, il existe une suite de puits $\beta'_{1}, \dots, \beta'_{k'}$ dans $(\Gamma,\widetilde \Lambda)$ telle que $\sigma_{\beta'_{1}}\dots\sigma_{\beta'_{k'}} \widetilde \Lambda = \Lambda'$. Finalement, on a que $\sigma_{\beta'_{1}}\dots\sigma_{\beta'_{k'}}\sigma_{\beta_{1}}\dots\sigma_{\beta_{k}}\Lambda =  \Lambda'$. 

Montrons maintenant l'existence d'une séquence $\beta_{1}, \dots, \beta_{k}$ de puits dans $(\Gamma,\Lambda)$ telle que $\sigma_{\beta_{1}}\dots\sigma_{\beta_{k}}\Lambda = \Lambda'$ lorsque $\Lambda$ et $\Lambda'$ sont deux orientations différents sur une seule arête $l$. Considérons le graphe $\Gamma \setminus l $ , il est scindé en deux composantes connexes car $\Gamma$ ne contient pas de cycle. Nommons $\Gamma_{1}$ la composante contenant le sommet $t(l)$, elle ne contient toujours pas de cycle, donc il existe un tri topologique de ces sommets $\gamma_{1}, \dots, \gamma_{m}$ pour l'orientation $\Lambda$. Montrons alors que $\sigma_{\gamma_{1}}\dots\sigma_{\gamma_{m}} \Lambda = \Lambda'$. En effet l'orientation des arêtes de $\Gamma_{1}$ différentes de $l$ est modifié deux fois donc reste inchangée. Alors que l'orientation de $l$ est modifiée une unique fois. D'après la propositon \ref{tri-topo}, $\gamma_{1}, \dots, \gamma_{m}$ est une suite de puit pour orientation $\Lambda$, donc on a le résultat.

Montrons maintenant le point 2, soit $\beta_{1}, \dots, \beta_{k}$ une suite de puits de $\Gamma$ tels que $\sigma_{\beta_{1}}\dots\sigma_{\beta_{k}}\Lambda =  \Lambda'$ et posons $\Psi^{+} = F^{+}_{\beta_{k}}F^{+}_{\beta_{k-1}} \dots F^{+}_{\beta_{1}}$ et $\Psi^{-} = F^{-}_{\beta_{1}}F^{-}_{\beta_{2}} \dots F^{-}_{\beta_{k}}$. Les foncteurs $\Psi^{+}$ et $\Psi^{-}$ sont respectivements définies de $\mathscr L(\Gamma,\Lambda)$ dans $\mathscr L(\Gamma, \Lambda')$ et $\mathscr L(\Gamma,\Lambda')$ dans $\mathscr L(\Gamma, \Lambda)$ d'après le point 1.
Si $V$ est une représentation de $\mathscr L(\Gamma,\Lambda)$ indécomposable et non isomorphe à $F^{-}_{\beta_{1}}F^{-}_{\beta_{2}} \dots F^{-}_{\beta_{i-1}}(L_{\beta_{i}})$ pour un certain $i \in [\![1,k]\!]$. D'après le corollaire \ref{cor-crucial}, on sait que $\Psi^{+}(V) \neq 0$ sinon il existe $i \in [\![1,k]\!]$ tel que $V$ soit isomorphe  à $F^{-}_{\beta_{1}}F^{-}_{\beta_{2}} \dots F^{-}_{\beta_{i-1}}(L_{\beta_{i}})$.
Donc en utilisant le théorème \ref{thm1}, on obtient que $\Psi^{+}(V)$ est indécomposable. Montrons que $\Psi^{+}(V)$ n'est pas isomorphe à $F^{+}_{\beta_{k}}F^{+}_{\beta_{k-1}} \dots F^{+}_{\beta_{j+1}}(L_{\beta_{j}})$ pour un certain $j\in [\![1,k]\!]$. Sinon on obtient que $F^{-}_{\beta_{j}} \dots F^{-}_{\beta_{k}}\Psi^{+}(V)$ isomorphe à $F^{+}_{\beta_{j-1}} \dots F^{+}_{\beta_{1}}(V)$ et isomorphe à $0$. On utilise à nouveau le corolaire \ref{cor-crucial} pour en déduire la contradiction $V$ isomorphe à $F^{-}_{\beta_{1}} \dots F^{-}_{\beta_{l-1}}(L_{\beta_{l}})$ pour un certain $l \in [\![1,k]\!]$. Finalement, on a montré que $\Psi^{+}(V)$ est indécomposable et non isomorphe à $F^{+}_{\beta_{k}}F^{+}_{\beta_{k-1}} \dots F^{+}_{\beta_{j+1}}(L_{\beta_{j}})$ pour un certain $j\in [\![1,k]\!]$, c'est à dire que $\Psi^{+}$ est bien défini entre $\mathscr M \setminus \widetilde{\mathscr M}$ et  $\mathscr M' \setminus \widetilde{\mathscr M'}$. De plus, on sait que $\Psi^{+}\Psi^{-}(V)$ est isomorphe à $V$ donc $\Psi^{+}$ est une bijection entre $\mathscr M \setminus \widetilde{\mathscr M}$ et  $\mathscr M' \setminus \widetilde{\mathscr M'}$.
%% peut-on parler de bijection quand on a pas forcément des ensembles ?
\end{proof}

\begin{defi}
	Soit $(\Gamma,\Lambda)$ un graphe orienté sans cycle orienté et $\gamma_{1}, \dots, \gamma_{n}$ un tri topologique sur les sommmet de $\Gamma$ pour l'orientation $\Lambda$. On apelle les \emph{foncteurs de Coxeter} les foncteurs $\Phi^{+} = F^{+}_{\beta_{k}}F^{+}_{\beta_{k-1}} \dots F^{+}_{\beta_{1}}$ et $\Phi^{-} = F^{-}_{\beta_{1}}F^{-}_{\beta_{2}} \dots F^{-}_{\beta_{k}}$.
\end{defi}

\begin{lm}
  \begin{enumerate}
  \item Les foncteurs $\Phi^{+}$ et $\Phi^{-}$ sont définies de $\mathscr(\Gamma, \Lambda)$ dans lui-même.
  \item Les foncteurs $\Phi^{+}$ et $\Phi^{-}$ ne sont dépendent du choix du tri topologique choisi.
  \end{enumerate}
\end{lm}
\begin{proof}
  Démontrons le point 1, soit $(\gamma_{1}, \dots, \gamma_{n})$ un tri topologique sur les sommets de $\Gamma$ pour l'orientation $\Lambda$, on remarque que l'orientation de chaque des arêtes de $\Gamma$ est modifiée par le foncteur $\Psi^{+}$ obtenu par ce tri topologique. Ainsi on a $\mathscr L(\Gamma,\sigma_{\gamma_{1}} \dots \sigma_{\gamma_{n}}\Lambda) = \mathscr L(\Gamma,\Lambda)$; c'est à dire que $\Psi^{+}$ envoie la catégorie $\mathscr L(\Gamma,\Lambda)$ dans elle-même.

  Passons au point 2, et remarquons d'abord que si $\delta_{1}$ et $\delta_{2}$ sont deux puits de $(\Gamma,\widetilde \Lambda)$ pour $\widetilde \Lambda$ une certaine orientation, non joints par une arête alors $F^{+}_{\delta_{1}}$ et $F^{+}_{\delta_{2}}$ commutent. 
Soit $\gamma_{1}, \dots, \gamma_{n}$ et $\gamma'_{1}, \dots , \gamma'_{n}$ deux tris topologiques sur les sommets de $\Gamma$ pour l'orientation $\Lambda$, et soit $m \in [\![1,n]\!]$ tel que $\gamma_{1}= \gamma'_{m}$. Montrons que $\gamma_{1} = \gamma'_{m}$ n'est joint à aucun des sommets $\gamma_{1},\dots, \gamma_{m-1}$. En effet, si $i \in [\![1,m-1]\!]$, il n'existe pas d'arête de $\gamma'_{i}$ vers $\gamma'_{m}$ car $i<m$ et si $j \in [\![2,n]\!]$ tel que $\gamma_{j} = \gamma'_{i}$ alors il n'existe pas d'arête de $\gamma_{j}$ vers $\gamma_{1}$ car $1<j$. Avec la remarque ci-dessus, on obtient $F_{\gamma'_{m}}$ et $F_{\gamma'_{i}}$ commutent pour $i \in [\![1,m-1]\!]$ et $F^{+}_{\gamma'_{m}}F^{+}_{\gamma'_{m-1}}\dots F^{+}_{\gamma'_{1}} = F^{+}_{\gamma'_{m-1}}\dots F^{+}_{\gamma'_{1}} F^{+}_{\gamma'_{m}} = F^{+}_{\gamma'_{m-1}}\dots F^{+}_{\gamma'_{1}}F^{+}_{\gamma_{1}}$. En utilisant sucessivement, le même argument pour $\gamma_{2}$ puis $\gamma_{3}$ et ainsi de suite. On obtient : 
\begin{eqnarray}%% je ne sais plus comment on supprime la numérotation
  F_{\gamma'_{n}}\dots F_{\gamma'_{1}} = F_{\gamma_{n}} \dots F_{\gamma_{1}}
\end{eqnarray}
\end{proof}

\begin{defi}
  Soit $(\Gamma,\Lambda)$ un graphe orienté sans cycle orienté. 
  \begin{enumerate}
  \item On dit qu'un objet $V \in \mathscr L(\Gamma,\Lambda)$ est \emph{(+)-irrégulier} (respectivement \emph{(-)-irrégulier}) si $(\Phi^{+})^{k}(V) = 0$ (respectivement $(\Phi^{-})^{k}(V)$)pour un certain $k$ entier.
  \item On dit qu'un objet $V \in \mathscr L(\Gamma,\Lambda)$ est \emph{régulier} si $V$ isomorphe à $(\Phi^{-})^{k}(\Phi^{+})^{k}(V)$ et à $(\Phi^{+})^{k}(\Phi^{-})^{k}(V)$ pour tout entier $k$
  \end{enumerate}
\end{defi}
\begin{thm}
  Soit $(\Gamma,\Lambda)$ un graphe orienté sans cycle orienté.
  \begin{enumerate}
  \item Chaque $V \in \mathscr L(\Gamma,\Lambda)$ indécomposable est soit régulier soit (+) ou (-)-irrégulier.
  \item Soit $\gamma_{1},\dots,\gamma_{n}$ un tri topologique des sommets de $\Gamma$ pour l'orientation $\Lambda$ et posons $V_{i} = F^{-}_{\gamma_{1}}F^{-}_{\gamma_{2}} \dots F^{-}_{\gamma_{i-1}}(L_{\gamma_{i}}) \in \mathscr L(\Gamma,\Lambda)$ et  $\widehat V_{i} = F^{+}_{\gamma_{1}}F^{+}_{\gamma_{2}} \dots F^{+}_{\gamma_{i-1}}(L_{\gamma_{i}}) \in \mathscr L(\Gamma,\Lambda)$ pour $i \in [\![1,n]\!]$. Alors $\Phi^{+}(V_{i})=0$ et chaque objet indécomposable $V \in \mathscr L(\Gamma,\Lambda)$ tel que $\Phi^{+}(V) =0$ est isomorphe à l'un des $V_{i}$. De même, si $\Phi^{-}(\widehat V_{i}) = 0$ et si $V$ est indécomposable avec $\Phi^{-}(V) = 0$, alors $V$ est isomorphe à l'un des $\widehat V_{i}$.
\item  Chaque objet $V$ (+)-irrégulier (respectivement (-)-irrégulier) indécomposable a la forme de $(\Phi^{-})^{k}(V_{i})$ (respectivement $(\Phi^{+})^{k}(\widehat V_{i})$) pour $i$ et $k$ entiers.
  \end{enumerate}
\end{thm}
\begin{proof}
  Démontrons le point 1, on remarque premièrement que $V\in \mathsrc L(\Gamma,\Lambda)$ irréductible ne peut être régulier et (+)- ou (-)-irrégulier. En effet, s'il existe un certain  $k$ tel que $(\Phi^{+})^{k}(V) = 0$ ou $(\Phi^{-})^{k}(V) = 0$ alors on ne peut pas avoir $V$, $(\Phi^{-})^{k}(\Phi^{+})^{k}(V)$ et $(\Phi^{+})^{k}(\Phi^{-})^{k}(V)$ isomorphes car l'un des deux derniers est nul et $V$ est indécomposable. Supposons $V$ est non (-)-irrégulier et non (+)-irregulier et $V$ indécomposable, montrons alors que $V$ est régulier. Comme $(\Phi^{+})^{k}(V)$ n'est non nul pour tout $k$ entier, on sait d'après le corollaire \ref{cor-crucial} que $(\Phi^{+})^{k}(V)$ est indécomposable pour tout $k$ entier. Et ainsi, en composant à gauche $k$ fois par $(\Phi^{-})$, on obtient par le théorème \ref{thm1} que $(\Phi^{-})^{k}(\Phi^{+})^{k}(V)$ est isomorphe à $V$. Comme $V$ est (-)-irrégulier et par le même raisonnement, on en déduit que $(\Phi^{+})^{k}(\Phi^{-})^{k}(V)$ est isomorphe à $V$, d'où le résultat espéré.
  Démontrons maintenant le point 2, les $V_{i}$ et les $\widehat{V}_{i}$ sont soit indécomposables, soit nul d'après le corolaire \ref{cor-crucial}. D'où, $\Phi^{+}(V_{i}) =0$ et $\Phi^{-}(\widehat{V}_{i})=0$, si $V_{i}$ et $\widehat{V}_{i}$ sont nuls. S'ils sont indécomposables, $\Phi^{+}(V_{i})$ et $\Phi^{+}(\widehat{V}_{i})$ sont respectivement isomophes à $F^{+}_{\gamma_{n}} \dots F^{+}_{\gamma_{i}}(l_{\gamma_{i}}$ et $F^{-}_{\gamma_{1}} \dots F^{-}_{\gamma_{i}}(L_{\gamma_{i}}$, tous deux isomorphes à 0. Par une utilisation directe du deuxième point du corolaire \ref{cor-crucial}, si $V$ est indécomposable, $\Phi^{+}(V) = 0$ implique que $V$ est isomorphe à l'un des $V_{i}$ et  $\Phi^{-}(V)=0$ implique que $V$ est isomorphe à l'un des $\widehat{V}_{i}$.
Démontrons le point 3,  soit $k$ l'entier minimal tel que $\Phi^{+-}^{k}(V) = 0$ on pose $W = \Phi^{+-}^{k-1}(V)$ est indécomposable et on utilise le point précédent.



 remarquons qu'aucun des $V_{i}$ et $\widehat V_{i}$ ne peut être nul, car pour tout $j$ entier strictement inférieur à $i$ $\gamma_{j}$ est différent de $\gamma_{i}$. Ainsi $(V_{i})_{\gamma_{i}} = (V)_{\gamma_{i}}$ d'après l'action des fonctions $F^{-}_{\gamma_{j}}$. De même, $\widehat V_{i}$ est non nul pour les même raisons. Donc les $V_{i}$ et $\widehat V_{i}$ sont tous indécomposables, et finalement .
\end{proof}
\clearpage


\section{Graphes, groupe de Weyl et transformation de Coxeter}
	Dans cette section nous définirons un système de racine utile, ainsi que des résultats qui nous permettrons dans la dernière partie de prouver le théorème de Gabriel.
		\begin{defi}
		Soit $\Gamma$ un graphe ne possédant pas de boucles.
		\begin{enumerate}
			\item On note $\mathscr E_{\Gamma}$ le $\mathbb Q$-espace vectoriel $\mathbb Q^{\Gamma_0}$ composé des familles de rationnels $x=(x_\gamma)_{\gamma\in\Gamma_0}$. On associe à chaque sommet $\gamma\in\Gamma_0$, le vecteur $\bar{\gamma}\in\mathscr E_\Gamma$ tel que $(\bar\gamma)_\gamma=1$ et $\forall \delta\neq\gamma,\;(\bar\gamma)_\delta=0$. On dit qu'un vecteur $x$ est \emph{entier} si $\forall\gamma\in\Gamma_{0},\;x_\gamma\in\mathbb Z$. On dit qu'un vecteur $x$ est \emph{positif}, et l'on note $x>0$, si $x\neq0$ et si $\forall\gamma\in\Gamma_0,\;x_\gamma\geq0$.
			\item On pose :
				\[
				\begin{array}{rccc}
					B:&\mathscr E_{\Gamma}&\longrightarrow&\mathbb Q\\
					&x&\longmapsto&\underset{\gamma\in\Gamma_0}{\sum} x_\gamma^2-\underset{l\in\Gamma_1}{\sum}x_{e_1(l)}x_{e_2(l)}\\
				\end{array}
			\]
			où $e_1(l)$ et $e_2(l)$ sont les extrémités de $l$.
		\item Pour chaque sommet $\gamma\in\Gamma_0$, on pose l'application $\sigma_\gamma$ de $\mathscr E_\Gamma$ à valeurs dans $\mathscr E_\Gamma$ définie par $\forall\delta\neq\gamma,\; (\sigma_\gamma x)_\delta=x_\delta$ et $(\sigma_\gamma x)\gamma=-x_\gamma+\sum_{l\in\Gamma^\gamma}x_{e(l)}$, où $e(l)$ est l'extrémité de $l$ différente de $\gamma$. On note $W$ l'ensemble engendré en composant les applications $\sigma_\gamma$.
		\end{enumerate}
		\end{defi}
		\begin{prop}
			L'application $B$ est une forme quadratique, dont on notera $\ps{}{}$ la forme bilinéaire symétrique associée. Pour tout sommet $\gamma\in\Gamma_0$, l'application $\sigma_\gamma$ est linéaire.
		\end{prop}
		\begin{proof}
			Soient $x,y\in\EG$, $\lambda\in\mathbb{Q}$, on remarque tout d'abord que $B(\lambda x)=\sum_{\gamma\in\Gamma_0}(\lambda x_\gamma)^2-\sum_{l\in\Gamma_1}\lambda x_{e_1(l)}\lambda x_{e_2(l)}=\lambda^2B(x)$. On pose $\ps{x}{y}=\frac{1}{2}(B(x+y)-B(x)-B(y))$, on a donc bien $\ps{x}{x}=\frac{1}{2}(B(2x)-2B(x))=B(x)$, et il vient :
			\[
		\begin{array}{>{\dps}r>{\dps}l>{\dps}l}
			2\ps{x}{y}&=& \underset{\gamma\in\Gamma_0}{\sum}(x+y)_\gamma^2 -\underset{l\in\Gamma_1}{\sum}(x+y)_{e_1(l)}(x+y)_{e_2(l)}\\
		&&-\underset{\gamma\in\Gamma_0}{\sum}x_\gamma^2 +\underset{l\in\Gamma_1}{\sum}x_{e_1(l)}x_{e_2(l)}-\underset{\gamma\in\Gamma_0}{\sum}y_\gamma^2 +\underset{l\in\Gamma_1}{\sum}y_{e_1(l)}y_{e_2(l)}\\
		\text{d'où }\ps{x}{y}&=&  \underset{\gamma\in\Gamma_0}{\sum}x_\gamma y_\gamma -\frac{1}{2}(\underset{l\in\Gamma_1}{\sum}x_{e_1(l)} y_{e_2(l)}+\underset{l\in\Gamma_1}{\sum}y_{e_1(l)} x_{e_2(l)}).
		\end{array}
			\]
			De cette dernière égalité on tire immédiatement :
			\begin{itemize}
					\item $\forall x,y\in\EG,\; \ps{x}{y}=\ps{y}{x}$
					\item $\forall x,y,z\in\EG,\; \ps{x+y}{z}=\ps{x}{z}+\ps{y}{z}$
					\item $\forall x,y\in\EG,\lambda\in\mathbb Q,\;\ps{\lambda x}{y}=\lambda\ps{x}{y}$
			\end{itemize}
			Ainsi, l'application $\ps{}{}$ est bilinéaire et symétrique, l'application $B$ est donc une forme quadratique. Prouvons maintenant la seconde partie de la proposition, à cet effet, soient $\gamma\in\Gamma_0$, $x,y\in\EG$ et $a,b\in\mathbb Q$. On a $\forall\delta\neq\gamma,\;(\sigma_\gamma(ax+by))_\delta=ax_\delta+by_\delta=a(\sigma_\gamma x)_\delta+b(\sigma_\gamma y)_\delta$ et $(\sigma_\gamma (ax+by))\gamma=-(ax_\gamma+by_\gamma)+\sum_{l\in\Gamma^\gamma}(ax_{e(l)+}+by_{e(l)})=a(\sigma_\gamma x)_\gamma+b(\sigma_\gamma y)_\gamma$. On a donc bien $\sigma_\gamma(ax+by)=a\sigma_\gamma x+b\sigma_\gamma y$, or le sommet $\gamma\in\Gamma_{0}$ est arbitraire, cela conclut donc la démonstration.
		\end{proof}
		\begin{lm}
			\label{elem}
		Soit $\Gamma$ un graphe sans boucle, $\gamma,\delta\in\Gamma_0$, $x\in\EG$ :
		\begin{enumerate}
			\item $\ps{\bar\gamma}{\bar\gamma}=1$, et si $\gamma\neq\delta$, alors $-2\ps{\bar\gamma}{\bar\delta}$ est le nombre d'arêtes liant $\gamma$ et $\delta$;
			\item on a $\sigma_\gamma(x)=x-2\ps{\bar\gamma}{x}\bar\gamma$ et $\sigma_\gamma^2=\Id$ ; en particulier $W$ est un groupe ;
			\item le groupe $W$ préserve les éléments entiers de $\EG$ et la forme quadratique $B$ ;
			\item si la forme $B$ est définie positive, alors le groupe $W$ est fini.
		\end{enumerate}
\end{lm}
		\begin{proof}
			Commençons par prouver le point $1$. Soient $\gamma,\delta\in\Gamma_0$, avec $\gamma\neq\delta$, alors $\ps{\bar\gamma}{\bar\gamma}=B(\bar\gamma)=1$, en effet la somme sur les arêtes vaut 0, et dans la somme sur les sommets, seul le terme en $\gamma$ est non nul et il vaut $1$. Calculons maintenant $\ps{\bar\gamma}{\bar\delta}$, comme $\gamma\neq\delta$, on a $\forall\varepsilon\in\Gamma_0,\;\bar\gamma_\varepsilon\bar\delta_\varepsilon=0$. Ainsi $-2\ps{\bar\gamma}{\bar\delta}=\sum_{l\in\Gamma_1}(\bar\gamma_{e_1(l)}\bar\delta_{e_2(l)}+\bar\gamma_{e_2(l)}\bar\delta_{e_1(l)})$. Notons $\Gamma^\gamma\cap\Gamma^\delta=\left\{ l_1,\dots,l_n \right\}$ les arêtes liant les sommets $\gamma$ et $\delta$. Alors $\forall l\notin\left\{ l_1,\dots,l_n \right\},\;\bar\gamma_{e_1(l)}=\bar\gamma_{e_2(l)}=0$ ou $\bar\delta_{e_1(l)}=\bar\delta_{e_2(l)}=0$, d'où $\bar\gamma_{e_1(l)}\bar\delta_{e_2(l)}+\bar\gamma_{e_2(l)}\bar\delta_{e_1(l)}=0$. Et, $\forall i \in\ent{1}{n}$, soit $\bar\gamma_{e_1(l_i)}\bar\delta_{e_2(l_i)}=1$ et $\bar\gamma_{e_2(l_i)}\bar\delta_{e_1(l_i)}=0$, soit $\bar\gamma_{e_1(l_i)}\bar\delta_{e_2(l_i)}=0$ et $\bar\gamma_{e_2(l_i)}\bar\delta_{e_1(l_i)}=1$. D'où :
			\[
		\begin{array}{>{\dps}r>{\dps}l>{\dps}l}
			-2\ps{\bar\gamma}{\bar\delta}&=& \sum_{l\in\Gamma_1}(\bar\gamma_{e_1(l)}\bar\delta_{e_2(l)}+\bar\gamma_{e_2(l)}\bar\delta_{e_1(l)})\\
			&=& \sum_{i=1}^n(\bar\gamma_{e_1(l_i)}\bar\delta_{e_2(l_i)}+\bar\gamma_{e_2(l_i)}\bar\delta_{e_1(l_i)})\\
			&=& \sum_{i=1}^n 1\\
			&=& n,
		\end{array}
			\]
			ce qui termine la démonstration du premier point, passons au deuxième. Soient $x\in\EG,\gamma\in\Gamma_0$, on a $x=\sum_{\delta\in\Gamma_0}x_\delta\bar\delta$. Ainsi on a $(\sigma_\gamma x)_\delta=x_\delta=(x-2\ps{\bar\gamma}{x}\bar\gamma)_\delta$, dès que $\gamma\neq\delta$ car $\bar\gamma_\delta=0$, et :
			\[
		\begin{array}{>{\dps}r>{\dps}l>{\dps}l}
			(x-2\ps{\bar\gamma}{x}\bar\gamma)_{\gamma}&=& x_\gamma-2\ps{\bar\gamma}{x}\\
			&=& x_\gamma-2\ps{\bar\gamma}{\sum_{\delta\in\Gamma_0}x_\delta\bar\delta}\\
			&=& x_\gamma-2\ps{\bar\gamma}{\bar\gamma}x_\gamma+\sum_{\delta\in\Gamma_0,\delta\neq\gamma}-2\ps{\bar\gamma}{\bar\delta}x_\delta\\
			&=& -x_\gamma+\sum_{\delta\in\Gamma,\delta\neq\gamma}\Card(\Gamma^\gamma\cap\Gamma^\delta) x_\delta\\
			&=& -x_\gamma+\sum_{\delta\in \mathscr V_\gamma}\sum_{l\in\Gamma^\delta\cap\Gamma^\gamma}x_{e(l)}\\
			&=& -x_\gamma+\sum_{l\in\Gamma^\gamma}x_{e(l)}\\
			&=& (\sigma_\gamma x)_\gamma
		\end{array}
			\]
			où $\mathscr V_\gamma$ est l'ensemble des sommets voisins de $\gamma$. Ainsi on a $\sigma_\gamma x=x-2\ps{\bar\gamma}{x}\bar\gamma$. Calculons maintenant $\sigma_\gamma^2$ :
			\[
		\begin{array}{>{\dps}r>{\dps}l>{\dps}l}
			\sigma_\gamma(\sigma_\gamma(x))&=& \sigma_\gamma(x)-2\ps{\bar\gamma}{\sigma_\gamma(x)}\bar\gamma\\
			&=& \sigma_\gamma(x)-2\ps{\bar\gamma}{x-2\ps{\bar\gamma}{x}\bar\gamma}\bar\gamma\\
			&=& x-2\ps{\bar\gamma}{x}\bar\gamma-2\ps{\bar\gamma}{x}\bar\gamma+4\ps{\bar\gamma}{x}\bar\gamma\\
			&=& x\\
		\end{array}
			\]
			d'où $\sigma_\gamma^2=\Id$. Ainsi $\Id\in W$, la composition est associative, et tous les éléments de $W$ sont inversibles. En effet soit $\omega\in W$, alors $\omega$ s'écrit $\omega=\prod_{i=1}^n \sigma_{\gamma_i}$ avec $\gamma_i\in\Gamma_0$ et ainsi $\omega'=\prod_{i=1}^n \sigma_{\gamma_{n+1-i}}$ est l'inverse de $\omega$, les termes s'annulant deux à deux. On a donc que $W$ est un groupe et ceci conclut la preuve du point $2$. Pour prouver le point $3$, soit $x\in\EG$ un vecteur entier et $\gamma\in\Gamma_0$ un sommet de $\Gamma$. On a $(\sigma_\gamma(x))_\delta=x_\delta\in\mathbb Z$ dès que $\delta\neq\gamma$ et $(\sigma_\gamma(x))_\gamma=-x_\gamma+\sum_{l\in\Gamma^\gamma}x_{e(l)}\in\mathbb Z$ car $\forall \delta\in\Gamma_0,\;x_\delta\in\mathbb Z$. Comme les éléments de $W$ sont des compositions des applications $\sigma_\gamma$, on a immédiatement que $W$ préserve les éléments entiers de $\EG$. Voyons maintenant que $W$ préserve $B$ :
			\[
		\begin{array}{>{\dps}r>{\dps}l>{\dps}l}
			B(\sigma_\gamma(x))&=& B(x-2\ps{\bar\gamma}{x}\bar\gamma)\\
			&=& B(x)+4\ps{\bar\gamma}{x}^2B(\bar\gamma)-2\ps{x}{2\ps{\bar\gamma}{x}\bar\gamma}\\
			&=& B(x).
		\end{array}
			\]
			Or $W$ est engendré par les applications $\sigma_\gamma$, on a donc que $W$ préserve la forme quadratique $B$. Enfin, prouvons le point $4$. À cet effet, on pose $B_1=\left\{ x\in\EG : x\text{ est entier et }B(x)=1 \right\}$. Supposons un instant que $B_1$ est un ensemble fini. D'après le point $3$, $W$ est un groupe qui préserve $B_1$, c'est donc un sous-groupe du groupe des permutations des éléments de $B_1$. $W$ est ainsi un sous-groupe d'un groupe fini donc $W$ est un groupe fini. Prouvons maintenant que $B_1$ est fini. On note $\left[ B \right]$ la matrice représentant la forme quadratique $B$. Cette dernière est symétrique, ainsi, d'après le théorème spectral, il existe une base orthonormée $\mathscr B$ dans laquelle $\left[ B \right]$ est diagonale, notons $\lambda_1,\dots,\lambda_n$ ses valeurs propres, elles sont strictement positives car $B$ est définie positive. Soit $v\in B_1$, on note $v_1,\dots,v_n$ ses coordonnées dans $\mathscr B$. On a donc $\lambda_1v_1^2+\dots+\lambda_nv_n^2=1$, d'où $\forall i \in\ent{1}{n},\;|v_i|\leq\frac{1}{\sqrt{\lambda_i}}$. Ainsi les coordonnées des élements de $B_1$ sont bornées dans la base $\mathscr B$, elle sont donc également bornées dans la base canonique de $\EG$, utilisée pour définir les éléments entiers. Ainsi $B_1$ est une sous-partie bornée de $\mathbb Z^n$ donc est finie. Or on a prouvé précédemment que cela implique que $W$ est fini, la preuve est donc complète.
		\end{proof}
		\begin{prop}
		La forme quadratique $B$ est définie positive si et seulement si $\Gamma$ est de la forme $A_n,D_n,E_6,E_7$ ou $E_8$.
		\end{prop}
		\begin{proof}
			Soit $\Gamma$ un graphe. On va d'abord prouver que si $B$ est définie positive, $\Gamma$ est nécessairement de la forme $A_n,D_n,E_6,E_7$ ou $E_8$. Tout d'abord, remarquons que $\Gamma$ ne peut contenir de sous-graphe d'une des formes suivantes : $\widetilde A_n$ ou $\widetilde D_n$, où $n$ représente le nombre de sommets. $\widetilde A_n$ est défini pour $n\geq 2$ et $\widetilde D_n$ est défini pour $n\geq 5$.


	\begin{tikzpicture}
    			\foreach \x in {0,...,5}
			\node[sommet] (\x) at (\x,0) {};
    			\foreach \z in {0,...,5}
			\node (\z') at (\z,-0.3cm) {$1$};
    			\foreach \y in {0,...,1}
			\draw[thick] (\y) -- +(1cm,0);
			\foreach \y in {3,...,4}
			\draw[thick] (\y) -- +(1cm,0);
			\draw[dotted, thick] (2) -- (3);
			\draw (0) to[bend left] (5);
			\node (t) at (2.5,-1) {$\widetilde A_n$};
	\end{tikzpicture}
	\begin{tikzpicture}
    			\foreach \x in {0,...,5}
			\node[sommet] (\x) at (\x,0) {};
    			\foreach \z in {0,...,5}
			\node (\z') at (\z,-0.3cm) {$2$};
    			\foreach \y in {0,...,1}
			\draw[thick] (\y) -- +(1cm,0);
			\foreach \y in {3,...,4}
			\draw[thick] (\y) -- +(1cm,0);
			\draw[dotted, thick] (2) -- (3);
			\foreach \x in {-.5,5.5}
			\foreach \y in {-0.5,.5}
			\node[sommet] (\x\y) at (\x,\y) {};
			\foreach \x in {-.5,5.5}
			\foreach \y in {-0.8,.2}
			\node (\x..\y) at (\x,\y) {$1$};
			\foreach \y in {-0.5,.5}
			\draw[thick] (-0.5,\y) to (0);
			\foreach \y in {-0.5,.5}
			\draw[thick] (5.5,\y) to (5);
			\node (t) at (2.5,-1) {$\widetilde D_n$};
	\end{tikzpicture}
	Pour cela, supposons par l'absurde que $\Gamma$ contient un sous-graphe de la forme $\widetilde A_n$ ou $\widetilde D_n$. Si $\Gamma$ contient un sous-graphe de la forme $\widetilde A_n$, on prend $x\in\EG$ tel que $x_\gamma=0$ lorsque $\gamma$ n'est pas un élément du sous-graphe $\widetilde A_n$, et $x_\gamma=1$ lorsque $\gamma$ appartient à $\widetilde A_n$. Alors, comme $\widetilde A_n$ contient $n$ sommets et $n$ arêtes, on a $B(x)=n-n=0$ alors que $x$ n'est pas nul. Donc $B$ n'est pas définie positive, ce qui est une contradiction. Supposons maintenant que $\Gamma$ contient un sous-graphe de la forme $\widetilde D_n$, on prend dans ce cas $x\in\EG$ tel que $x_\gamma=0$ dès que $\gamma\notin\widetilde D_n$, $x_\gamma=1$ si $\gamma\in\widetilde D_n$ possède un seul voisin dans $\widetilde D_n$, ou de manière équivalente si $\gamma$ est une des quatre extrémités de $\widetilde D_n$, et $x_\gamma=2$ si $\gamma\in\widetilde D_n$ possède au moins deux voisins dans $\widetilde D_n$, c'est-à-dire si $\gamma$ est un point non extrémal de $\widetilde D_n$. Comme $\widetilde D_n$ possède $n-4$ sommets non extrémaux, de valeurs $2$ dans $x$, $4$ extrémités, de valeurs $1$ dans $x$, la somme sur les sommets dans la définition de $B(x)$ vaut $4(n-4)+4\times1=4(n-3)$. D'autre part $\widetilde D_n$ contient $n-5$ arêtes centrales (c'est-à-dire dont aucune des extrémités de l'arête n'est un point extrémal), et $4$ arêtes extrémales (ou non centrales), ainsi la somme sur les arêtes dans la définition de $B(x)$ vaut $4\times2+4(n-5)=4(n-3)$. D'où $B(x)=4(n-3)-4(n-3)=0$ alors que $x$ est non nul, donc $B$ n'est pas définie positive, ce qui est une contradiction. Ainsi, si $B$ est définie positive, $\Gamma$ est nécessairement de la forme :
	\begin{equation}
		\label{dessin}
	\begin{tikzpicture}
    			\foreach \x in {0,...,10}
			\node[sommet] (\x) at (\x,0) {};
    			\foreach \x in {11,...,16}
			\node[sommet] (\x) at (\x-6,1) {};
    			\foreach \y in {0,1,3,4,5,6,8,9,11,12,14,15}
			\draw[thick] (\y) -- +(1cm,0);
			\draw[thick] (5) -- +(0,1cm);
			\foreach \x in {2,7,13}
			\draw[dotted, thick] (\x) -- +(1cm,0);
			\node (a) at (5cm,-0.3cm) {$a$};
    			\foreach \x in {1,...,3}
			\node () at (\x-1,-.3cm) {$x_\x$};
			\node () at (3,-.3cm) {$x_{p-1}$};
			\node () at (4,-.3cm) {$x_{p}$};
			\node () at (6,-.3cm) {$x_{q}$};
			\node () at (7,-.3cm) {$y_{q-1}$};
			\node () at (8,-.3cm) {$y_{3}$};
			\node () at (9,-.3cm) {$y_{2}$};
			\node () at (10,-.3cm) {$y_{1}$};
			\node () at (5,1.3cm) {$z_{r}$};
			\node () at (6,1.3cm) {$z_{r-1}$};
			\node () at (7,1.3cm) {$z_{r-2}$};
			\node () at (8,1.3cm) {$z_{3}$};
			\node () at (9,1.3cm) {$z_{2}$};
			\node () at (10,1.3cm) {$z_{1}$};
	\end{tikzpicture}
	\end{equation}
	où $p,q,r\in\mathbb N$ sont des entiers positifs. Posons maintenant, pour un certain entier $p\in\mathbb N$, l'application $C_p$ en $p+1$ variables :
		\[
			C_p(x_1,\dots,c_{p+1})=-x_1x_2-x_2x_3-\dots-x_px_{p+1}+x_1^2+\dots+x_p^2+\frac{p}{2(p+1)}x_{p+1}^2,
	\]
	et prouvons que $C_p$ est une forme quadratique positive, de noyau de dimension $1$ et qui vérifie que si $x\neq0$ est tel que $C_p(x)=0$ alors $x$ n'a aucune coordonnée nulle. Pour prouver tout cela, nous allons démontrer que 
	\begin{equation}
		C_p(x)=\sum_{i=1}^p\frac{i}{2(i+1)}(x_{i+1}-\frac{i+1}{i}x_i)^2.
		\label{Cp}
	\end{equation}
	Faisons le calcul :
	\[
		\begin{array}{>{\dps}r>{\dps}l>{\dps}l}
	\sum_{i=1}^p\frac{i}{2(i+1)}(x_{i+1}-\frac{i+1}{i}x_i)^2&=& \sum_{i=1}^p\frac{i}{2(i+1)}(x_{i+1}^2-2\frac{i+1}{i}x_ix_{i+1}+(\frac{i+1}{i}x_i)^2)\\
	&=& -\sum_{i=1}^px_ix_{i+1}+\sum_{i=1}^p\frac{i}{2(i+1)}x_{i+1}^2+\sum_{i=1}^p\frac{i+1}{2i}x_i^2\\
	&=& -\sum_{i=1}^px_ix_{i+1}+\frac{p}{2(p+1)}x_{p+1}^2+x_1^2+\sum_{i=2}^p\frac{(i-1)+(i+1)}{2i}x_i^2\\
	&=& C_p(x),\\
\end{array}
	\]
	ainsi on a bien l'égalité annoncée plus haut. Gr\^ace à cette nouvelle écriture (\ref{Cp}), on a immédiatement la positivité de $C_p$. Soit $x\in\EG$ tel que $C_p(x)=0$, d'après (\ref{Cp}), on a donc $\forall i \in\ent{1}{n},\;x_{i+1}=\frac{i+1}{i}x_i$, d'où $\forall i\in\ent{1}{n},\;x_i=ix_1$ par récurrence. Ainsi $\Ker C_p\subset\Vect\left\{ (1,2,\dots,n) \right\}$, et la réciproque est immédiate. Donc $\Ker C_p=\Vect\left\{ (1,2,\dots,n) \right\}$, et ainsi le noyau de $C_p$ est de dimension $1$. De plus, comme $(1,2,\dots,n)$ ne possède pas de coordonnée nulle, $x\neq0$ et $x\in\Vect\left\{ (1,2,\dots,n) \right\}$ implique $\forall i\in\ent{1}{n},\;x_i\neq0$. Donc si $x\neq0$ et $C_p(x)=0$, alors toutes les coordonnées de $x$ sont non nulles. Enfin prouvons que $C_p$ est une forme quadratique, pour cela posons :
	\[
		\ps{x}{y}_p=\sum_{i=1}^p\frac{i}{2(i+1)}(x_{i+1}-\frac{i+1}{i}x_i)(y_{i+1}-\frac{i+1}{i}y_i),
	\]
on a ainsi :
\begin{enumerate}
	\item $\forall x\in\EG$, $\ps{x}{x}_p=C_p(x)$
	\item $\forall x,y\in\EG$, $\ps{x}{y}=\ps{y}{x}$
	\item $\forall x,y,z\in\EG$, $\ps{x+y}{z}=\ps{x}{z}+\ps{y}{z}$
	\item $\forall x,y\in\EG,\lambda\in\mathbb Q$, $\ps{\lambda x}{y}=\lambda\ps{x}{y}$,
\end{enumerate}
ce qui prouve que $C_p$ est bien une forme quadratique. Et on a ainsi prouvé toutes les propriétés annoncées à propos de $C_p$. On place maintenant les nombres $x_1,\dots,x_p,y_1,\dots,y_q,z_1,\dots,z_r$ et $a$ sur le graphe $\Gamma$ comme représenté sur le schéma (\ref{dessin}), formant ainsi un vecteur $t\in\EG$. On a alors :
\[
\begin{array}{>{\dps}r>{\dps}l>{\dps}l}
	B(t)&=& \sum_{\gamma\in\Gamma_0}t_\gamma^2-\sum_{l\in\Gamma_1}t_{e_1(l)}t_{e_2(l)}\\
	&=& x_1^2+\dots+x_p^2+y_1^2+\dots+y_q^2+\dots+z_1^2+\dots+z_r^2+a^2\\
	&& -x_1x_2-\dots-x_pa-y_1y_2-\dots-y_qa-\dots-z_1z_2-\dots-z_ra\\
	&=& C_p(x_1,\dots,x_p,a)+C_q(y_1,\dots,y_q,a)+C_r(z_1,\dots,z_r,a)\\
	&& +\left(1-\frac{p}{2(p+1)}-\frac{q}{2(q+1)}-\frac{r}{2(r+1)}\right)a^2.\\
\end{array}
\]
Gr\^ace à cette dernière formule, on prouve que $B$ est définie postivie si et seulement si $\frac{p}{2(p+1)}+\frac{q}{2(q+1)}+\frac{r}{2(r+1)}<1$. Supposons que $\frac{p}{2(p+1)}+\frac{q}{2(q+1)}+\frac{r}{2(r+1)}\geq1$, alors on prends $a>0$, $x_i=\frac{i}{p+1}a$, $y_i=\frac{i}{q+1}a$ et $z_i=\frac{i}{r+1}a$, ce qui nous donne que $(x_1,\dots,x_p,a)\in\Ker C_p$, $(y_1,\dots,y_q,a)\in\Ker C_q$, $(z_1,\dots,z_r,a)\in\Ker C_r$, d'où $B(t)\leq0$ et ainsi $B$ n'est pas définie positive. Supposons maintenant que $\frac{p}{2(p+1)}+\frac{q}{2(q+1)}+\frac{r}{2(r+1)}<1$, et qu'on a $B(t)=0$. Comme $B(t)$ est une somme de termes positifs, ils doivent tous être nulles, on obtient ainsi $a=0$, puis, d'après les propriétés des noyaux de $C_p,C_q,C_r$, si l'un des termes est non nul, ils le sont tous, or $a=0$, on a donc $x_1=\dots= x_p=y_1=\dots=y_q=z_1=\dots=z_r=0$. On vient de voir que $B(t)=0$ implique $t=0$ donc $B$ est définie positive. Ainsi $B$ est définie postivie si et seulement si $\frac{p}{2(p+1)}+\frac{q}{2(q+1)}+\frac{r}{2(r+1)}<1$. Cette inégalité équivaut en fait à $\frac{1}{p+1}+\frac{1}{q+1}+\frac{1}{r+1}>1$, en effet les inégalités suivantes sont équivalentes :
\[
\begin{array}{>{\dps}r>{\dps}l>{\dps}l}
	\frac{p}{2(p+1)}+\frac{q}{2(q+1)}+\frac{r}{2(r+1)}&<&1\\
	\cfrac{\frac{2(p+1)}{2}-1}{2(p+1)}+\cfrac{\frac{2(q+1)}{2}-1}{2(q+1)}+\cfrac{\frac{2(r+1)}{2}-1}{2(r+1)}&<&1\\
	\frac{3}{2}-1&<&\frac{1}{2(p+1)}+\frac{1}{2(q+1)}+\frac{1}{2(r+1)}\\
	1&<&\frac{1}{p+1}+\frac{1}{q+1}+\frac{1}{r+1}\\
\end{array}
\]
On peut supposer, quitte à échanger les rôles de $p,q,r$, que $p\leq q\leq r$. On pose $A=\frac{1}{p+1}+\frac{1}{q+1}+\frac{1}{r+1}$ et on étudie si $A>1$ (c'est-à-dire si $B$ est définie positive) :
\begin{enumerate}
	\item $p=0$, $q$ et $r$ sont arbitraires, alors $A>1$ et on est dans le cas $A_n$ ;
	\item $p=1$, $q=1$, et $r$ est arbitraire, alors $A>1$ et on est dans le cas $D_n$ ;
	\item $p=1$, $q=2$ et $r=2,3,4$, alors $A>1$ et on est dans les cas $E_6,E_7,E_8$ ;
	\item $p=1$, $q=2$, et $r\geq5$, alors $A\leq1$ ;
	\item $p=1$, $q\geq3$ et $r\geq3$ alors $A\leq1$ ;
	\item $p\geq2$, $q\geq2$ et $r\geq2$ alors $A\leq1$ ;
\end{enumerate}
On a ainsi énuméré tous les cas possibles pour $p,q,r$ et les seuls graphes pour lequels $A>1$ sont les graphes de type $A_n,D_n,E_6,E_7$ et $E_8$. Donc la forme quadratique $B$ est définie positives si et seulement si $\Gamma$ est de type $A_n,D_n,E_6,E_7$ ou $E_8$.
\end{proof}
\begin{defi}
	Soit $\Gamma$ un graphe. Un vecteur $x\in\EG$ est appelé \emph{racine} s'il existe $\gamma\in\Gamma_0$, $w\in W$, tel que $x=w(\bar\gamma)$. Les vecteurs $\bar\gamma$, où $\gamma\in\Gamma_0$, sont appelés \emph{racines simples}. Une racine $x$ est dite \emph{positive} si $x>0$.
\end{defi}
\begin{lm}
	\label{lmroot}
	Soit $\Gamma$ un graphe et $x\in\EG$ un vecteur.
	\begin{enumerate}
		\item Si $x$ est une racine, $x$ est entier et $B(x)=1$ ;
		\item si $x$ est une racine, $-x$ est une racine ;
		\item si $x$ est une racine, soit $x>0$, soit $-x>0$.
	\end{enumerate}
\end{lm}
\begin{proof}
	Soit $x\in\EG$ une racine, il existe alors $\gamma\in\Gamma_0$, $w\in W$, tel que $x=w(\bar\gamma)$. $B(\bar\gamma)=1$ et $\bar\gamma$ est entier, ainsi d'après le point $3$ du lemme \ref{elem}, $B(w(\bar\gamma))=B(x)=1$ et $w(\bar\gamma)=x$ est entier, ce qui prouve le point $1$. Le point $2$ suit du fait que $\forall\gamma\in\Gamma_0$, $\sigma_\gamma(\bar\gamma)=-\bar\gamma$, ainsi $w(\sigma_\gamma(\bar\gamma))=-w(\bar\gamma)=-x$, et comme $w\sigma_\gamma\in W$, $-x$ est une racine. On peut écrire la racine $x$ sous la forme $\sigma_{\gamma_1}\sigma_{\gamma_2}\dots\sigma_{\gamma_k}(\bar\gamma)$, il est donc suffisant, pour prouver le point $3$, de vérifier que si $y\in\EG$ est une racine positive, et si $\gamma\in\Gamma_0$ est un sommet quelconque de $\Gamma$, alors $\sigma_\gamma(y)$ est une racine et soit $\sigma_\gamma(y)>0$, soit $-\sigma_\gamma(y)>0$. Par définition, si $y$ est une racine, on a que $\sigma_\gamma(y)$ est une racine. Comme $B(y)=1$ (d'après le point $1$), et $B(\bar\gamma)=1$, on sait d'après l'inégalité de Cauchy-Schwarz que $|\ps{\bar\gamma}{y}|\leq1$. De plus, si on écrit $y=\sum_{i=1}^ny_{i}\bar\gamma_i$, on a $2\ps{\bar\gamma}{y}=2\sum_{i=1}^ny_i\ps{\bar\gamma}{\bar\gamma_i}$, où $y_i,\ps{\bar\gamma}{\bar\gamma_i}\in\mathbb Z$, car $y$ est une racine donc est entier d'après le point $1$, et $\ps{\bar\gamma}{\bar\gamma_i}$ est entier d'après le lemme \ref{elem}, d'où $2\ps{\bar\gamma}{y}\in\mathbb Z$. Ainsi on a 5 valeurs possibles pour $2\ps{\bar\gamma}{y}$ : $-2,-1,0,1,2$. Regardons les différents cas possibles :
	\begin{itemize}
		\item si $2\ps{\bar\gamma}{y}\leq0$, alors $\sigma_\gamma(y)=y-2\ps{\bar\gamma}{y}\bar\gamma$ est positif car $y$ est positif et $-2\ps{\bar\gamma}{y}\bar\gamma$ est positif ou nul ;
		\item si $2\ps{\bar\gamma}{y}=1$, alors $1=2\ps{\bar\gamma}{y}=2y_\gamma-\sum_{l\in\Gamma^\gamma}y_{e(l)}$ et ainsi $y_\gamma>0$ (sinon $2\ps{\bar\gamma}{y}\leq0$) d'où $y_\gamma\geq1$ comme $y$ est entier. Ainsi, $\sigma_\gamma(y)=y-\bar\gamma$ est positif (on ne peut pas avoir $y=\bar\gamma$ car sinon $2\ps{\bar\gamma}{y}=2$) ;
		\item si $2\ps{\bar\gamma}{y}=2$, alors $\ps{\bar\gamma-y}{\bar\gamma-y}=\ps{\bar\gamma}{\bar\gamma}-2\ps{\bar\gamma}{y}+\ps{y}{y}=1-2+1=0$, ainsi, si $B$ est définie positive, on conclue que $\bar\gamma-y=0$, donc $y=\bar\gamma$ et $-\sigma_\gamma(y)=-(-\gamma)=\gamma>0$.
	\end{itemize}
	On a ainsi prouvé le point $3$ dans le cas où $B$ est définie positive, et on se contentera de cette démonstration car on utilisera ce résultat uniquement lorsque $B$ sera définie positive. Pour une preuve dans un cas plus général, on pourra regarder dans \cite{S66}. 
\end{proof}
\begin{defi}
	Soit $\Gamma$ un graphe sans boucle, et soit $\gamma_1,\dots,\gamma_n$ une énumération de ses sommets. Un élément $c=\sigma_{\gamma_n}\dots\sigma_{\gamma_1}\in W$ est appelé \emph{transformation de Coxeter}, $c$ dépend de l'énumération choisie.
\end{defi}
\begin{lm}
Soit $\Gamma$ un graphe sans boucle, $\gamma_1,\dots,\gamma_n$ une énumération de ses sommets, et supposons que la forme quadratique $B$ est définie positive pour ce graphe $\Gamma$. On note $c=\sigma_{\gamma_n}\dots\sigma_{\gamma_1}$.
\begin{enumerate}
	\item La transformation de Coxeter $c$ n'admet aucun point fixe non nul dans $\EG$ ;
	\item si $x\in\EG$ et $x\neq0$, alors il existe $i\in\mathbb N$ tel que $c^i(x)$ n'est pas positif.
\end{enumerate}
\end{lm}
\begin{proof}
	Supposons qu'il existe $y\in\EG$ avec $y\neq0$ et $c(y)=y$. Comme les applications $\sigma_{\gamma_n}$, $\sigma_{\gamma_{n-1}},\dots,\sigma_{\gamma_2}$ ne changent pas la coordonnée en $\gamma_1$ (c'est-à-dire $\forall i\in\ent{2}{n}\;\forall z\in\EG,\;(\sigma_{\gamma_i}(z))_{\gamma_1}=z_{\gamma_1}$), on a $(\sigma_{\gamma_1}(y))_{\gamma_1}=c(y)_{\gamma_1}=y_{\gamma_1}$. Comme $\sigma_{\gamma_1}$ laisse les coordonnées différentes de $\gamma_1$ inchangées, on a $\sigma_{\gamma_1}(y)=y$. Puis, par récurrence, on prouve que $\forall i\in\ent{1}{n}$, $\sigma_{\gamma_i}(y)=y$. De plus, on a $\forall\gamma\in\Gamma_0$, $\sigma_{\gamma}(y)=y=y-2\ps{\bar\gamma}{y}\bar\gamma$, d'après le point $2$ du lemme \ref{elem}, d'où $\forall\gamma\in\Gamma_0$, $\ps{\bar\gamma}{y}=0$. Les $\bar\gamma$ formant une base de $\EG$, cela nous donne $\forall x\in\EG$, $\ps{x}{y}=0$, ainsi $y=0$, ce qui est une contradiction. Le point $1$ est ainsi prouvé, passons au point $2$. Comme $B$ est définie positive, on sait d'après le point $4$ du lemme \ref{elem} que le groupe $W$ est finie, ainsi il existe $h\in\mathbb N$ tel que $c^h=\Id$. Soit $x\in\EG$, avec $x\neq0$, et supposons par l'absurde que $\forall i\in\mathbb N$, $c^i(x)>0$, alors les vecteurs $x,c(x),\dots,c^{h-1}(x)$ sont positifs, donc le vecteur $y=x+c(x)+\dots+c^{h-1}(x)$ est positif. Or $c(y)=c(x)+c^2(x)+\dots+c^{h-1}(x)+c^h(x)$, et $c^h(x)=x$, donc $c(y)=y$ et $y$ est non nul car positif, ce qui contredit le point $1$. Le point $2$ est ainsi prouvé.
\end{proof}
\clearpage
\section{Le théorème de Gabriel}
Soit $(\Gamma,\Lambda)$ un carquois.
\begin{defi}[vecteur dimension]
	Pour chaque objet $V\in\mathscr L(\Gamma,\Lambda)$, on note $\dim V\in\EG$ le vecteur $(\dim V_\alpha)_{\alpha\in\Gamma_0}$, qu'on appellera \emph{vecteur dimension}.
\end{defi}
%Lignes à ajouter pour faire apparaître notre bibliographie
\clearpage
\bibliographystyle{unsrt}
\bibliography{biblio}
\end{document}
